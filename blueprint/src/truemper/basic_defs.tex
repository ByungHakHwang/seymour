\section{Basic Definitions}

\subsection{Matroid Structure}

\begin{definition}[matroid]
  \label{def:matroid}
  The \emph{matroid} $M$ over ground set $E$ is the pair $(E,\mathcal{I})$ where $I \subseteq \mathcal{P}(E)$ satisfying the following axioms:
  \begin{enumerate}[label=(\roman*)]
    \item The null set $\varnothing$ is in $\mathcal{I}$.
    \item Every subset of any set in $\mathcal{I}$ is an element of $\mathcal{I}$.
    \item For any subset $\overline{E} \subseteq E$, the maximal subsets of $\overline{E}$ that are in $\mathcal{I}$ have the same cardinality.
  \end{enumerate}
  We call $\mathcal{I}$ collection of \emph{independent sets} of $E$, and refer to all other subsets of $E$ as \emph{dependent}.
\end{definition}

\begin{definition}[isomorphism]
  \label{def:isomorphism}
  Two matroids are isomorphic if they become equal upon a suitable relabeling of the elements.
\end{definition}

\begin{definition}[loop]
  \label{def:loop}
  Let $M = (E, \mathcal{I})$ be a matroid. A \emph{loop} is any element $x \in E$ such that $\{x\} \notin \mathcal{I}$.
  That is, a loop is any element $x \in E$ such that $\{x\}$ is a depenedent subset of $E$.
\end{definition}

\begin{definition}[coloop]
  \label{def:coloop}
  \todo[inline]{todo: add definition}
\end{definition}

\begin{definition}[parallel elements]
  \label{def:parallel_elems}
  \todo[inline]{todo: add definition}
\end{definition}

\begin{definition}[series elements]
  \label{def:series_elems}
  \todo[inline]{todo: add definition}
\end{definition}


\subsection{Matroid Classes}

\begin{definition}[binary matroid]
  \label{def:binary_matroid}
  Suppose that $F$ is a binary matrix $F$ whose columns are indexed by a set $E$.
  Let $\mathcal{I}$ be the collection of all subsets $Z \subseteq E$ such that the columns of $F$ indexed by $Z$ are linearly
  independent over $\GFtwo$. The pair $M = (E, \mathcal{I})$ is then called an \emph{binary matriod}.
\end{definition}

\begin{definition}[regular matroid]
  \label{def:regular_matroid}
  \uses{def:binary_matroid,def:tu_matrix}
  A binary matroid $M$ is regular if some binary representation matrix $B$ of $M$ has a totally unimodular signing (i.e., assignment of signs to the $1$s in $B$ that results in a TU matrix).
\end{definition}

% todo: Two of many important properties of regular matroids:
% First, a binary matroid M is regular if and only if every representation matrix B can be signed to become a real totally unimodular matrix.
% Second, every graphic or cographic matroid is regular.

\begin{definition}[graphic matroid]
  \label{def:graphic_matroid}
  Let $G$ be a graph and $E$ to be the set containing all of its edges. Define $\mathcal{I}$ to be the collection of all trees
  in $G$. Then the \emph{graphic matroid} generated by $G$ is the pair $M(G) = (E,\mathcal{I})$.
\end{definition}

\begin{definition}[cographic matroid]
  \label{def:cographic_matroid}
  \todo[inline]{todo: add definition}
\end{definition}

\begin{definition}[planar matroid]
  \label{def:planar_matroid}
  \todo[inline]{todo: add definition}
\end{definition}

\begin{definition}[dual matroid]
  \label{def:dual_matroid}
  \todo[inline]{todo: add definition}
\end{definition}

\begin{definition}[self-dual matroid]
  \label{def:self_dual_matroid}
  \todo[inline]{todo: add definition}
\end{definition}


\subsection{Specific Matroids (Constructions)}

\begin{definition}[wheel]
  \label{def:wheel}
  \todo[inline]{todo: add definition}
\end{definition}

\begin{definition}[$W_{3}$]
  \label{def:M_W_3}
  \todo[inline]{todo: add definition}
\end{definition}

\begin{definition}[$W_{4}$]
  \label{def:M_W_4}
  \todo[inline]{todo: add definition}
\end{definition}

\begin{definition}[$R_{10}$]
  \label{def:R10}
  \todo[inline]{todo: add definition}
\end{definition}

\begin{definition}[$R_{12}$]
  \label{def:R12}
  \todo[inline]{todo: add definition}
\end{definition}

\begin{definition}[$F_{7}$]
  \label{def:F7}
  \todo[inline]{todo: add definition}
\end{definition}

\begin{definition}[$F_{7}^{*}$]
  \label{def:F7_dual}
  \todo[inline]{todo: add definition}
\end{definition}

\begin{definition}[$M(K_{3,3})$]
  \label{def:M_K_3_3}
  \todo[inline]{todo: add definition}
\end{definition}

\begin{definition}[$M(K_{3,3})^{*}$]
  \label{def:M_K_3_3_dual}
  \todo[inline]{todo: add definition}
\end{definition}

\begin{definition}[$M(K_{5})$]
  \label{def:M_K_5}
  \todo[inline]{todo: add definition}
\end{definition}

\begin{definition}[$M(K_{5})^{*}$]
  \label{def:M_K_5_dual}
  \todo[inline]{todo: add definition}
\end{definition}


\subsection{Connectivity and Separation}

\begin{definition}[$k$-separation]
  \label{def:k_sep}
  % \uses{prop:3.3.18}
  See text after Proposition 3.3.18.
\end{definition}

\begin{definition}[$k$-connectivity]
  \label{def:k_conn}
  % \uses{prop:3.3.18}
  See text after Proposition 3.3.18.
\end{definition}


\subsection{Reductions}

\begin{definition}[deletion]
  \label{def:deletion}
  \todo[inline]{todo: add definition}
\end{definition}

\begin{definition}[contraction]
  \label{def:contraction}
  \todo[inline]{todo: add definition}
\end{definition}

% reduction = deletion or contraction

\begin{definition}[minor]
  \label{def:minor}
  \todo[inline]{todo: add definition}
\end{definition}


\subsection{Extensions}

% todo: see text on page 62

% todo: extension = addition or expansion, inverse to resp. deletion and contraction

\begin{definition}[$1$-element addition]
  \label{def:1_elem_addition}
  % \uses{}
  \todo[inline]{add name, label, uses, text}
\end{definition}

\begin{definition}[$1$-element expansion]
  \label{def:1_elem_expansion}
  % \uses{}
  \todo[inline]{add name, label, uses, text}
\end{definition}

\begin{definition}[$1$-element extension]
  \label{def:1_elem_ext}
  \todo[inline]{todo: add definition}
\end{definition}

\begin{definition}[$2$-element extension]
  \label{def:2_elem_ext}
  \todo[inline]{todo: add definition}
\end{definition}

\begin{definition}[$3$-element extension]
  \label{def:3_elem_ext}
  \todo[inline]{todo: add definition}
\end{definition}


\subsection{Sums}

\begin{definition}[$1$-sum]
  \label{def:1_sum}
  \todo[inline]{todo: add definition}
\end{definition}

\begin{definition}[$2$-sum]
  \label{def:2_sum}
  \todo[inline]{todo: add definition}
\end{definition}

\begin{definition}[$3$-sum]
  \label{def:3_sum}
  \todo[inline]{todo: add definition}
\end{definition}

\begin{definition}[$\Delta$-sum]
  \label{def:Delta_sum}
  \todo[inline]{todo: add definition}
\end{definition}

\begin{definition}[$Y$-sum]
  \label{def:Y_sum}
  \todo[inline]{todo: add definition}
\end{definition}


\subsection{Total Unimodularity}

\begin{definition}[TU matrix]
  \label{def:tu_matrix}
  A rational matrix $A$ is totally unimodular if every square submatrix $D$ of $A$ has $\det_{\Q} D = 0$ or $\pm 1$.
\end{definition}

% todo: In particular, all entries of a totally unimodular matrix must be $0$ or $\pm 1$.


\subsection{Auxiliary Results}

\begin{theorem}[Menger's theorem]
  \label{thm:Menger}
  A connected graph $G$ is vertex $k$-connected if and only if every two nodes are connected by $k$ internally node-disjoint paths.
  Equivalent is the following statement. $G$ is vertex $k$-connected if and only if any $m \leq k$ nodes are joined to any $n \leq k$ nodes by $k$ internally node-disjoint paths.
  One may demand that the $m$ nodes are disjoint from the $n$ nodes, but need not do so.
  Also, the $k$ paths can be so chosen that each of the specified nodes is an endpoint of at least one of the paths.
\end{theorem}

\begin{definition}[$\Delta Y$ exchange]
  \label{def:Delta_Y_exchange}
  \todo[inline]{add}
\end{definition}

\begin{definition}[gap]
  \label{def:gap}
  \todo[inline]{add}
\end{definition}
