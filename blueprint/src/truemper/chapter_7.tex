\section{Chapter 7}

\subsection{Chapter 7.2}

\begin{definition}[splitter]
  \label{def:splitter}
  \uses{def:binary_matroid,def:minor,def:isomorphism,def:k_conn}
  Let $\M$ be a class of binary matroids closed under isomorphism and under taking minors. Let $N$ be a $3$-connected minor of $\M$ on at least $6$ elements.
  If every $M \in \M$ with a proper $N$ minor has a $2$-separation, then $N$ is called a splitter of $\M$.
\end{definition}

\begin{theorem}[7.2.1.a splitter for nonwheels]
  \label{thm:7.2.1.a}
  \uses{def:binary_matroid,def:minor,def:isomorphism,def:k_conn,def:wheel,def:splitter}
  Let $\M$ be a class of binary matroids closed under isomorphism and under taking minors. Let $N$ be a $3$-connected minor of $\M$ on at least $6$ elements.
  If $N$ is not a wheel, then $N$ is a splitter of $\M$ iff $\M$ does not contain a $3$-connected $1$-element extension of $N$.
\end{theorem}

\begin{proof}[Proof sketch]
  \uses{thm:6.4.1,def:splitter,def:k_conn,def:1_elem_ext,def:minor,def:k_sep}
  \begin{itemize}
    \item If $N$ is a splitter of $\M$, then clearly $\M$ does not contain a $3$-connected $1$-element extension of $N$.
    \item Prove the converse by contradiction. To this end, suppose that $\M$ does not contain a $3$-connected $1$-element extension of $N$ and that $N$ is not a splitter of $\M$.
    \item Thus, $\M$ contains a $3$-connected matroid $M$ with a proper $N$ minor and no $2$-separation.
    \item Since $\M$ is closed under isomorphism, we may assume $N$ itself to be that $N$ minor.
    \item By Theorem 6.4.1 (applied to $M$ and $N$), $M$ has a $3$-connected minor $N'$ that is a $3$-connected $1$- or $2$-element extension of an $N$ minor.
    \item The $1$-extension case has been ruled out.
    \item In the $2$-element extension case, $N'$ is derived from the $N$ minor by one addition and one expansion. Again, since $\M$ is closed under isomorphism and minor taking, we may take $N$ itself to be that $N$ minor. Thus, $N'$ is derived from $N$ by one addition and one expansion.
    \item Let $C$ be a binary matrix representing $N'$ and displaying $N$. By investigating the structure of $C$, one can show that $N'$ contains a $3$-connected $1$-element extension of an $N$ minor, which has been ruled out.
  \end{itemize}
\end{proof}

\begin{theorem}[7.2.1.b splitter for wheels]
  \label{thm:7.2.1.b}
  \uses{def:binary_matroid,def:minor,def:isomorphism,def:k_conn,def:wheel,def:splitter}
  Let $\M$ be a class of binary matroids closed under isomorphism and under taking minors. Let $N$ be a $3$-connected minor of $\M$ on at least $6$ elements.
  If $N$ is a wheel, then $N$ is a splitter of $\M$ iff $\M$ does not contain a $3$-connected $1$-element extension of $N$ and does not contain the next larger wheel.
\end{theorem}

\begin{proof}[Proof sketch]
  \uses{thm:6.4.1,def:splitter,def:k_conn,def:1_elem_ext,def:minor,def:k_sep}
  Similar to proof of Theorem 7.2.1.a. The analysis of the matrix $C$ can be done in one go for both cases.
\end{proof}

\begin{corollary}[7.2.10.a]
  \label{cor:7.2.10.a}
  \uses{thm:7.2.1.a}
  Theorem 7.2.1.a specialized to graphs.
\end{corollary}

\begin{proof}[Proof sketch]
  \uses{thm:7.2.1.a}
  Consider the corresponding graphic matroids, apply splitter theorem, extensions in graphic matroids correspond to extensions in graphs.
\end{proof}

\begin{corollary}[7.2.10.b]
  \label{cor:7.2.10.b}
  \uses{thm:7.2.1.b}
  Theorem 7.2.1.b specialized to graphs.
\end{corollary}

\begin{proof}[Proof sketch]
  \uses{thm:7.2.1.b}
  Consider the corresponding graphic matroids, apply splitter theorem, extensions in graphic matroids correspond to extensions in graphs.
\end{proof}

\begin{theorem}[7.2.11.a]
  \label{thm:7.2.11.a}
  \uses{def:M_K_5,def:M_K_3_3,def:splitter,def:minor,def:graphic_matroid}
  $K_{5}$ is a splitter of the graphs without $K_{3,3}$ minors.
\end{theorem}

\begin{proof}[Proof sketch]
  \uses{cor:7.2.10.a,def:k_conn,def:1_elem_ext}
  Up to isomorphism, there is just one $3$-connected $1$-edge extension of $K_{5}$. To obtain it, one partitions one vertex of $K_{5}$ into two vertices of degree $2$ and connects the two vertices by a new edge. The resulting graph has a $K_{3,3}$ minor. Thus, the theorem follows from Corollary 7.2.10.a.
\end{proof}

\begin{theorem}[7.2.11.b]
  \label{thm:7.2.11.b}
  \uses{def:M_W_3,def:M_W_4,def:splitter,def:minor,def:graphic_matroid}
  $W_{3}$ is a splitter of the graphs without $W_{4}$ minors.
\end{theorem}

\begin{proof}[Proof sketch]
  \uses{cor:7.2.10.b,def:k_conn,def:1_elem_ext}
  There is no $3$-connected $1$-edge extension of $W_{3}$, so the theorem follows from Corollary 7.2.10.b.
\end{proof}


\subsection{Chapter 7.3}

\begin{theorem}[7.3.1.a]
  \label{thm:7.3.1.a}
  \uses{def:k_conn,def:binary_matroid,def:minor,def:wheel,def:isomorphism,def:gap}
  Let $M$ be a $3$-connected binary matroid with a $3$-connected proper minor $N$ on at least $6$ elements. Assume $N$ is not a wheel.
  Then for some $t \geq 1$, there is a sequence $M_{0}, \dots, M_{t} = M$ of nested $3$-connected minors where $M_{0}$ is isomorphic to $N$ and where the gap is $1$.
\end{theorem}

\begin{proof}[Proof sketch]
  \uses{thm:7.2.1.a}
  \begin{itemize}
    \item Inductively for $i \geq 0$ assume the existence of a sequence $M_{0}, \dots, M_{i}$ of $3$-connected minors where $M_{0}$ is isomorphic to $N$, $M_{i}$ is not a wheel, and the gap is $1$.
    \item If $M_{i} = M$, we are done, so assume that $M_{i}$ is a proper minor of $M$.
    \item Use the contrapositive of the splitter Theorem 7.2.1.a to find a larger sequence.
    \begin{itemize}
      \item Let $\M$ be the collection of all matroids isomorphic to a (not necessarily proper) minor of $M$.
      \item Since $M_{i}$ is a $3$-connected proper minor of the $3$-conected $M \in \M$, it cannot be a splitter of $\M$. By Theorem 7.2.1.a, $\M$ contains a matroid $M_{i + 1}$ that is a $3$-connected $1$-element extension of a matroid isomorphic to $M_{i}$.
      \item Since every $1$-element reduction of a wheel with at least $6$ elements is $2$-separable, $M_{i + 1}$ is not a wheel, as otherwise $M_{i}$ is $2$-separable, which is a contradiction.
    \end{itemize}
    \item If necessary, relabel $M_{0}, \dots, M_{i}$ so that they consistute a sequence of nested minors of $M_{i + 1}$. This sequence satisfies the induction hypothesis.
    \item By induction, the claimed sequence exists for $M$.
  \end{itemize}
\end{proof}

\begin{theorem}[7.3.1.b]
  \label{thm:7.3.1.b}
  \uses{def:k_conn,def:binary_matroid,def:minor,def:wheel,def:isomorphism,def:gap}
  Let $M$ be a $3$-connected binary matroid with a $3$-connected proper minor $N$ on at least $6$ elements. Assume $N$ is a wheel.
  Then for some $t \geq 1$, there is a sequence $M_{0}, \dots, M_{t} = M$ of nested $3$-connected minors where:
  \begin{itemize}
    \item $M_{0}$ is isomorphic to $N$,
    \item for some $0 \leq s \leq t$ the subsequence $M_{0}, \dots, M_{s}$ consists of wheels and has gap 2,
    \item the subsequence $M_{s}, \dots, M_{t}$ has gap $1$.
  \end{itemize}
\end{theorem}

\begin{proof}[Proof sketch]
  \uses{thm:7.2.1.b}
  Same as the proof of Theorem 7.3.1.a, but uses Theorem 7.2.1.b instead of 7.2.1.a to extend the sequence of minors.
\end{proof}

\begin{proposition}[7.2.1 from 7.3.1]
  \label{prop:7.2.1_from_7.3.1}
  \uses{thm:7.3.1.a,thm:7.3.1.b,thm:7.2.1.a,thm:7.2.1.b}
  Theorem 7.3.1 implies Theorem 7.2.1.
\end{proposition}

\begin{proof}[Proof sketch]
  \uses{thm:7.3.1.a,thm:7.3.1.b,thm:7.2.1.a,thm:7.2.1.b}
  \begin{itemize}
    \item Let $\M$ and $N$ be as specified in Theorem 7.2.1. Suppose $N$ is not a wheel.
    \item Prove the nontrivial ``if'' part by contradiction: let $M$ be a $3$-connected matroid of $\M$ with $N$ as a proper minor.
    \item By Theorem 7.3.1, there is a sequence $M_{0}, \dots, M_{t} = M$ of nested $3$-connected minors where $M_{0}$ is isomorphic to $N$ and where the gap is $1$.
    \item Since $\M$ is closed under isomorphism, we may assume that $M$ is chosen such that $M_{0} = N$.
    \item Then $M_{1} \in \M$ is a $3$-connected $1$-element extension of $N$, which contradicts the assumed absence of such extensions.
    \item If $N$ is a wheel, the proof is analogous.
  \end{itemize}
\end{proof}

\begin{corollary}[7.3.2.a]
  \label{cor:7.3.2.a}
  Let $G$ be a $3$-connected graph with a $3$-connected proper minor $H$ with at least $6$ edges. Assume $H$ is not a wheel. Then for some $t \geq 1$, there is a sequence of nested $3$-connected minors $G_{0}, \dots, G_{t} = G$ where $G_{0}$ is isomorphic to $H$, and where each $G_{i + 1}$ has exactly one edge beyond those of $G_{i}$.
\end{corollary}

\begin{proof}[Proof sketch]
  \uses{thm:7.3.1.a}
  Translate Theorem 7.3.1.a directly into graph language.
\end{proof}

\begin{corollary}[7.3.2.b]
  \label{cor:7.3.2.b}
  Let $G$ be a $3$-connected graph with a $3$-connected proper minor $H$ with at least $6$ edges. Assume $H$ is a wheel. Then for some $t \geq 1$, there is a sequence of nested $3$-connected minors $G_{0}, \dots, G_{t} = G$ where:
  \begin{itemize}
    \item $G_{0}$ is isomorphic to $H$,
    \item for some $0 \leq s \leq t$ the subsequence $G_{0}, \dots, G_{t}$ consists of wheels where each $G_{i + 1}$ has exactly one additional spoke beyond those of $G_{i}$,
    \item in the subsequence $G_{s}, \dots, G_{t}$ each $G_{i + 1}$ has exactly one edge beyond those of $G_{i}$.
  \end{itemize}
\end{corollary}

\begin{proof}[Proof sketch]
  \uses{thm:7.3.1.b}
  Translate Theorem 7.3.1.b directly into graph language.
\end{proof}

\begin{theorem}[7.3.3, wheel theorem]
  \label{thm:7.3.3}
  Let $G$ be a $3$-connected graph on at least $6$ edges. If $G$ is not a wheel, then $G$ has some edge $z$ such that at least one of the minors $G / z$ and $G \setminus z$ is $3$-connected.
\end{theorem}

\begin{proof}[Proof sketch]
  \uses{cor:5.2.15,cor:7.3.2.b}
  \begin{itemize}
    \item By Corollary 5.2.15, $G$ has a $W_{3}$ minor.
    \item Let $H$ be a largest wheel minor of $G$. Since $G$ is not a wheel, $H$ is a proper minor of $G$.
    \item Apply Corollary 7.3.2.b to $G$ and $H$ to get a sequence of nested $3$-connected minors $G_{0}, \dots, G_{t} = G$ where $G_{0}$ is isomorphic to $H$.
    \item Since $H$ is the largest wheel minor and $G$ is not a wheel, Corollary 7.3.2.b shows that $s = 0$ and $t \geq 1$.
    \item Additionally, from corollary we know that $G = G_{t}$ has exactly one extra edge compared to $G_{t - 1}$. In other words, $G_{t - 1} = G / z$ or $G \setminus z$ for some edge $z$.
  \end{itemize}
\end{proof}

\begin{theorem}[7.3.3 for binary matroids]
  \label{thm:7.3.3.binary}
  \uses{thm:7.3.1.a,thm:7.3.1.b}
  Theorem 7.3.3 can be rewritten for binary matroids instead of graphs.
\end{theorem}

\begin{proof}[Proof sketch]
  \uses{thm:7.3.1.a,thm:7.3.1.b}
  Similar to the proof of Theorem 7.3.3, but use Theorem 7.3.1 instead of Corollary 7.3.2.
\end{proof}

\begin{proposition}[7.3.4.observation]
  \label{prop:7.3.4.obs}
  \uses{thm:7.3.1.a,thm:7.3.1.b,lem:3.3.12,lem:6.2.6}
  Oservation in text on pages 160--161.
\end{proposition}

\begin{theorem}[7.3.4]
  \label{thm:7.3.4}
  \uses{def:k_conn,def:binary_matroid,def:minor,def:1_elem_expansion,def:1_elem_addition,def:1_elem_expansion}
  Let $M$ be a $3$-connected binary matroid with a $3$-connected proper minor $N$ on at least $6$ elements. If $M$ does not contain a $3$-connected $1$-element expansion (resp. addition) of any $N$ minor, then $M$ has a sequence of nested $3$-connected minors $M_{0}, \dots, M_{t} = M$ where $M_{0}$ is an $N$ minor of $M$ and where each $M_{i + 1}$ is obtained from $M_{i}$ by expansions (resp. additions) involving some series (resp. parallel) elements, possibly none, followed by a $1$-element addition (resp. expansion).
\end{theorem}

\begin{proof}[Proof sketch]
  \uses{thm:7.3.1.a,thm:7.3.1.b,prop:7.3.4.obs}
  \begin{itemize}
    \item The case in parenthesis is dual to the normally stated one. Thus, only consider expansions below.
    \item Apply construction from observation before Theorem 7.3.4 to the sequence of minors from Theorem 7.3.1 to get the desired sequence.
  \end{itemize}
\end{proof}

\begin{corollary}[7.3.5]
  \label{cor:7.3.5}
  \uses{thm:7.3.4}
  Specializes Theorem 7.3.4 to graphs.
\end{corollary}


\subsection{Chapter 7.4}

\begin{theorem}[7.4.1 planarity characterization]
  \label{thm:7.4.1}
  \uses{def:planar_matroid,def:M_K_3_3,def:M_K_5}
  A graph is planar if and only if it has no $K_{3,3}$ or $K_{5}$ minors.
\end{theorem}

\begin{proof}[Proof sketch]
  \uses{lem:3.2.48,cor:5.2.15,cor:7.3.5}
  \begin{itemize}
    \item "Only if": planarity is preserved by taking minors, and by Lemma 3.2.48 both $K_{3,3}$ and $K_{5}$ are not planar.
    \item Let $G$ be a connected nonplanar graph with all proper minors planar. Goal: show that $G$ is isomorphic to $K_{3,3}$ or $K_{5}$.
    \item Prove that $G$ cannot be $1$- or $2$-separable. Thus $G$ is $3$-connected.
    \item By Corollary 5.2.15, $G$ has a $W_{3}$ minor, say $H$. Note: no $H$ minor of $G$ can be extended to a minor of $G$ by addition of an edge that connects two nonadjacent nodes.
    \item Then by Corollary 7.3.5.b, there exists a sequence $G_{0}, \dots, G_{t} = G$ of $3$-connected minors where $G_{0}$ is an $H$ minor and $G_{i + 1}$ is constructed from $G_{i}$ following very specific steps.
    \item By minimality, $G_{t - 1}$ is planar and $G$ is not. Argue about a planar drawing of $G_{t - 1}$ and how $G$ can be derived from it. Show that this must result in a subdivision of $K_{3,3}$ or $K_{5}$.
  \end{itemize}
\end{proof}

\begin{theorem}[Kuratowski]
  \label{thm:Kuratowski}
  \uses{def:planar_matroid,def:M_K_3_3,def:M_K_5}
  A graph is planar if and only if it has no subdivision of $K_{3,3}$ or $K_{5}$.
\end{theorem}

\begin{proof}
  \uses{thm:7.4.1}
  Note: Theorem 7.4.1 is equivalent to Kuratowski's theorem: a $K_{3,3}$ minor induces a subdivision of $K_{3,3}$ and a $K_{5}$ minor also leads to a subdivision of $K_{5}$ or $K_{3,3}$ (the latter in the case when an expansion step splits a vertex of degree $4$ into two vertices of degree $3$ after the new edge is inserted).
\end{proof}
