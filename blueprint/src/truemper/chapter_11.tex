\section{Chapter 11}

\subsection{Chapter 11.2}

The goal of this chapter is to prove the ``simple'' direction of the regular matroid decomposition theorem.

Ingredients from Section 9.2: % todo: move ingredients to respective sections, add them as ``uses'' clauses
\begin{itemize}
  \item A matrix is TU if all its subdeterminants are $0$, $\pm 1$.
  \item A binary matroid is regular if it has a signing that is TU.
  \item By Lemma 9.2.6 and Corollary 9.2.7, this signing is unique up to scaling by $\pm 1$ factors.
  \item The signing can be accomplished by signing one arbitrarily selected row or column at a time.
\end{itemize}

Ingredients from minimal violation matrices: % todo: move ingredients to respective sections, add them as ``uses'' clauses
\begin{itemize}
  \item Definition: a minimal violation matrix of total unimodularity (minimal violation matrix, MVM) is a $\{0, \pm 1\}$ matrix that is not TU, but all its submatrices are TU.
  \item MVMs are square and have determinant not equal to $0, \pm 1$.
  \item In particular, a $2 \times 2$ violation matrix has four $\pm 1$'s.
  \item Cosider a MVM of order $\ge 3$. Perform a pivot in it, then delete the pivot row and column. Then the resulting matrix is also MVM ("by a simple cofactor argument").
  % Note: pivots are defined in Chapter 2.3, page 20. Also in Chapter 3.2, page 34 for graphic matroids.
\end{itemize}

\begin{lemma}[11.2.1]
  \label{lem:11.2.1}
  \uses{def:regular_matroid,def:1_sum,def:2_sum}
  Any $1$- or $2$-sum of two regular matroids is also regular.
\end{lemma}

\begin{proof}[Proof sketch]
  \uses{prop:8.2.1,prop:8.2.3,prop:8.2.4}
  \begin{itemize}
    \item $1$-sum case: $M_{1} \oplus_{1} M_{2}$ is represented by a matrix $B = \mathrm{diag} (A_{1}, A_{2})$ where $A_{1}$ and $A_{2}$ represent $M_{1}$ and $M_{2}$. Use the same signings for $A_{1}$ and $A_{2}$ in $B$ to prove that $B$ is TU and hence the $1$-sum is regular.
    \item $2$-sum case: Slightly more complicated signing process. Similarly, reuse signings from $M_{1}$ and $M_{2}$, define signing on remaining nonzero elements via a concrete formula, then prove that the resulting matrix is TU.
  \end{itemize}
\end{proof}

\begin{lemma}[11.2.7]
  \label{lem:11.2.7}
  \uses{prop:8.3.10,prop:8.3.11,prop:8.5.3}
  $M_{2}$ of (8.3.10) and (8.3.11) is regular iff $M_{2\Delta}$ of (8.5.3) ($M_{2}$ converted by a $\Delta Y$ exchange) is regular.
\end{lemma}

\begin{proof}[Proof sketch]
  Utilize signings, minimal violation matrices, intersections (inside matrices), column dependence, pivot, duality.
\end{proof}

\begin{corollary}[11.2.8]
  \label{cor:11.2.8}
  \uses{def:Delta_Y_exchange,def:regular_matroid}
  $\Delta Y$ exchanges maintain regularity.
\end{corollary}

\begin{proof}
  \uses{lem:11.2.7}
  Follows by Lemma 11.2.7.
\end{proof}

\begin{lemma}[11.2.9]
  \label{lem:11.2.9}
  \uses{def:regular_matroid,def:3_sum}
  Any $3$-sum of two regular matroids is also regular.
\end{lemma}

\begin{proof}[Proof sketch]
  \uses{lem:11.2.7,cor:11.2.8}
  Yet more complicated, but similar. Uses the result that ``$\Delta Y$ exchanges maintain regularity'' (Corollary 11.2.8 of Lemma 11.2.7). The rest of the arguments are similar to the $2$-sum case: prove that submatrices are TU, then prove that the whole matrix is TU.
\end{proof}

\begin{theorem}[11.2.10]
  \label{thm:11.2.10}
  \uses{def:regular_matroid,def:1_sum,def:2_sum,def:3_sum}
  Any $1$-, $2$-, or $3$-sum of two regular matroids is regular.
\end{theorem}

\begin{proof}[Proof sketch]
  \uses{lem:11.2.1,lem:11.2.9}
  Combine Lemmas 11.2.1 and 11.2.9.
\end{proof}

\begin{corollary}[11.2.12]
  \label{cor:11.2.12}
  \uses{def:regular_matroid,def:Delta_sum,def:Y_sum}
  Any $\Delta$-sum or $Y$-sum of two regular matroids is also regular.
\end{corollary}

\begin{proof}[Proof sketch]
  \uses{def:Delta_sum,def:Y_sum,thm:11.2.10,cor:11.2.8}
  Follows from definitions of $\Delta$-sums and $Y$-sum, together with Theorem 11.2.10 and Corollary 11.2.8.
\end{proof}


\subsection{Chapter 11.3}

\begin{proposition}[11.3.3]
  \label{prop:11.3.3}
  \uses{prop:10.2.8}
  Graph plus $T$ set representing $R_{10}$
\end{proposition}

\begin{proposition}[11.3.5]
  \label{prop:11.3.5}
  \uses{prop:10.2.4}
  Graph plus $T$ set representing $F_{7}$.
\end{proposition}

\begin{proposition}[11.3.11]
  \label{prop:11.3.11}
  \uses{prop:9.2.14}
  The binary representation matrix $B^{12}$ for $R_{12}$.
\end{proposition}

The goal of the chapter is to prove the ``hard'' direction of the regular matroid decomposition theorem.

\begin{theorem}[11.3.2]
  \label{thm:11.3.2}
  \uses{def:regular_matroid,def:R10,def:splitter}
  $R_{10}$ is a splitter of the class of regular matroids.

  In short: up to isomorphism, the only $3$-connected regular matroid with $R_{10}$ minor is $R_{10}$.
\end{theorem}

\begin{proof}[Proof sketch]
  \uses{thm:7.2.1.a,prop:11.3.3,def:isomorphism,def:1_elem_ext,prop:11.3.5,def:F7,def:contraction} % thm:11.3.1 = thm:7.2.1
  \begin{itemize}
    \item Splitter theorem case (a)
    \item $R_{10}$ is self-dual, so it suffices to consider $1$-element additions.
    \item Represent $R_{10}$ by (11.3.3)
    \item Up to isomorphism, there are only $3$ distinct $3$-connected $1$-element extensions.
    \item Case 1 (graphic): contract a certain edge, the resulting graph contains a subdivision of (11.3.5), which represents $F_{7}$. Thus, this extension is nonregular.
    \item Cases 2, 3 (nongraphic): reduce instances to (11.3.5), same conclusion.
  \end{itemize}
\end{proof}

\begin{theorem}[11.3.10]
  \label{thm:11.3.10}
  \uses{cor:6.3.24,def:R12}
  In short: Restatement of \ref{cor:6.3.24} for $R_{12}$.
  Replacements: $\M$ is the class of regular matroids, $N$ is $R_{12}$, (6.3.12) is (11.3.6), (6.3.21-23) are (11.3.7-9).
\end{theorem}

\begin{theorem}[11.3.12]
  \label{thm:11.3.12}
  \uses{def:regular_matroid,def:R12,def:minor,def:k_sep,prop:11.3.11,def:isomorphism}
  Let $M$ be a regular matroid with $R_{12}$ minor. Then any $3$-separation of that minor corresponding to the $3$-separation $(X_{1} \cup Y_{1}, X_{2} \cup Y_{2})$ of $R_{12}$ (see (11.3.11) -- matrix $B^{12}$ for $R_{12}$ defining the $3$-separation) under one of the isomorphisms induces a $3$-separation of $M$.

  In short: every regular matroid with $R_{12}$ minor is a $3$-sum of two proper minors.
\end{theorem}

\begin{proof}[Proof sketch]
  \uses{def:1_elem_ext,prop:10.2.9,thm:11.3.10}
  \begin{itemize}
    \item Preparation: calculate all $3$-connected regular $1$-element additions of $R_{12}$. This involves somewhat tedious case checking. (Representation of $R_{12}$ in (10.2.9) helps a lot.) By the symmetry of $B^{12}$ and thus by duality, this effectively gives all $3$-connected $1$-element extensions as well.
    \item Verify conditions of theorem 11.3.10 (which implies the result).
    \item (11.3.7) and (11.3.9) are ruled out immediately from preparatory calculations.
    \item The rest is case checking ((c.1) and (c.2)), simpified by preparatory calculations.
  \end{itemize}
\end{proof}

\begin{theorem}[11.3.14 regular matroid decomposition, easy direction]
  \label{thm:11.3.14.seymour_easy}
  \uses{def:regular_matroid,def:graphic_matroid,def:cographic_matroid,def:isomorphism,def:R10,def:1_sum,def:2_sum,def:3_sum}
  Every binary matroid produced from graphic, cographic, and matroids isomorphic to $R_{10}$ by repeated $1$-, $2$-, and $3$-sum compositions is regular.
\end{theorem}

\begin{proof}[Proof sketch]
  \uses{thm:11.2.10}
  Follows from theorem 11.2.10.
\end{proof}

\begin{theorem}[11.3.14 regular matroid decomposition, hard direction]
  \label{thm:11.3.14.seymour_hard}
  \uses{def:regular_matroid,def:graphic_matroid,def:cographic_matroid,def:isomorphism,def:R10,def:R12,def:1_sum,def:2_sum,def:3_sum,def:k_conn,def:k_sep,prop:11.3.11}
  Every regular matroid $M$ can be decomposed into graphic and cographic matroids and matroids isomorphic to $R_{10}$ by repeated $1$-, $2$-, and $3$- sum decompositions.
  Specifically: If $M$ is a regular $3$-connected matroid that is not graphic and not cographic, then $M$ is isomorphic to $R_{10}$ or has an $R_{12}$ minor. In the latter case, any $3$-separation of that minor corresponding to the 3-separation $(X_{1} \cup Y_{1}, X_{2} \cup Y_{2})$ of $R_{12}$ ((11.3.11)) under one of the isomorphisms induces a $3$-separation of $M$.
\end{theorem}

\begin{proof}[Proof sketch]
  \uses{thm:10.4.1.if,thm:11.3.2,thm:11.3.12,lem:8.3.12}
  \begin{itemize}
    \item Let $M$ be a regular matroid. Assume $M$ is not graphic and not cographic.
    \item If $M$ is $1$-separable, then it is a $1$-sum. If $M$ is $2$-separable, then it is a $2$-sum. Thus assume $M$ is $3$-connected.
    \item By theorem 10.4.1, $M$ has an $R_{10}$ or an $R_{12}$ minor.
    \item $R_{10}$ case: by theorem 11.3.2, $M$ is isomorphic to $R_{10}$.
    \item $R_{12}$ case: by theorem 11.3.12, $M$ has an induced by $3$-separation, so by lemma 8.3.12, $M$ is a $3$-sum.
  \end{itemize}
\end{proof}


\subsection{Extensions of Regular Matroid Decomposition}

\begin{itemize}
  \item Theorem 11.3.14 remains valid when $3$-sums are replaced by $\Delta$- and $Y$-sums (Theorem 11.3.16).
  \item Theorem 11.3.14 (and 11.3.16) can also be proved for matroids with no $F_{7}$ minors or with no $F_{7}^{*}$ minors. (Uses Lemma 11.3.19: $F_{7}$ ($F_{7}^{*}$) is a splitter of the binary matroids with no $F_{7}^{*}$ ($F_{7}$) minors.)
\end{itemize}


\subsection{Applications of Regular Matroid Decomposition}

\begin{itemize}
  \item Efficient algorithm for testing if a binary matroid is regular (Section 11.4).
  \item Efficient algorithm for deciding if a rational matrix is TU (Section 11.4).
  \item Constructing TU matrices (Theorem 11.5.9). (Translate $3$-sum version of theorem 11.3.16 into matrix language.)
  \item Constructing ${0, 1}$ TU matrices (Theorem 11.5.13).
  \item Characterization of the cycle polytope (theorem 11.5.17). (Problem: let $M$ be a connected binary matroid with ground set $E$ and element weighs $w_{e}$ for all $e \in E$. Find a disjoint union $C$ of circuits of $M$ such that $\sum_{e \in C} w_{e}$ is maximized.)
  \item Number of nonzeros in TU matrices (Theorem 11.5.18).
  \item Triples in circuits (Theorem 11.5.18).
  \item Odd cycles (Theorem 11.5.20).
\end{itemize}
