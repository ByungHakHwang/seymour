\section{Chapter 8}

\subsection{Chapter 8.2}

This chapter is about deducing and manipulating $1$- and $2$-sum decompositions and compositions.

\begin{proposition}[8.2.1] % todo: this defines 1-sum
  \label{prop:8.2.1}
  \uses{def:k_sep}
  Matrix of $1$-separation.
\end{proposition}

\begin{lemma}[8.2.2]
  \label{lem:8.2.2}
  \uses{def:binary_matroid,def:1_sum,def:graphic_matroid,def:planar_matroid}
  Let $M$ be a binary matroid. Assume $M$ to be a $1$-sum of two matroids $M_{1}$ and $M_{2}$.
  \begin{itemize}
    \item If $M$ is graphic, then there eixst graphs $G$, $G_{1}$, $G_{2}$ for $M$, $M_{1}$, $M_{2}$, respectively, such that identification of a node of $G_{1}$ with one of $G_{2}$ creates $G$.
    \item If $M_{1}$ and $M_{2}$ are graphic (resp. planar), then $M$ is graphic (resp. planar).
  \end{itemize}
\end{lemma}

\begin{proof}[Proof sketch]
  \uses{thm:3.2.25.a}
  Elementary application of Theorem 3.2.25.a.
\end{proof}

\begin{proposition}[8.2.3]
  \label{prop:8.2.3}
  \uses{def:k_sep}
  Matrix of exact $2$-separation.
\end{proposition}

\begin{proposition}[8.2.4] % todo: this defines 2-sum
  \label{prop:8.2.4}
  \uses{prop:8.2.3}
  Matrices $B^{1}$ and $B^{2}$ of $2$-sum.
\end{proposition}

\begin{lemma}[8.2.6]
  \label{lem:8.2.6}
  \uses{def:binary_matroid,def:k_sep,def:k_conn,def:2_sum}
  Any $2$-separation of a connected binary matroid $M$ produces a $2$-sum with connected components $M_{1}$ and $M_{2}$.
  Conversely, any $2$-sum of two connected binary matroids $M_{1}$ and $M_{2}$ is a connected binary matroid $M$.
\end{lemma}

\begin{proof}[Proof sketch]
  \uses{prop:8.2.3,prop:8.2.4,lem:3.3.19}
  \begin{itemize}
    \item Definitions imply everything except connectedness.
    \item It is easy to check that connectedness of (8.2.3) implies connectedness of (8.2.4) and vice versa.
    \item By Lemma 3.3.19, connectedness of representation matrices is equivalent to connectedness of the corresponding matroids.
  \end{itemize}
\end{proof}

\begin{lemma}[8.2.7]
  \label{lem:8.2.7}
  \uses{def:binary_matroid,def:k_conn,def:2_sum,prop:8.2.3,prop:8.2.4,def:graphic_matroid,def:planar_matroid}
  Let $M$ be a connected binary matroid that is a $2$-sum of $M_{1}$ and $M_{2}$, as given via $B$, $B_1$, and $B_2$ of (8.2.3) and (8.2.4).
  \begin{itemize}
    \item If $M$ is graphic, then there exist $2$-connected graphs $G$, $G_{1}$, and $G_{2}$ for $M$, $M_{1}$, and $M_{2}$, respectively, with the following feature. The graph $G$ is produced when one identifies the edge $x$ of $G_{1}$ with the edge $y$ of $G_{2}$, and when subsequently the edge so created is deleted.
    \item If $M_{1}$ and $M_{2}$ are graphic (resp. planar), then $M$ is graphic (resp. planar).
  \end{itemize}
\end{lemma}

\begin{proof}[Proof sketch]
  \uses{def:k_sep,thm:3.2.25.b,lem:8.2.6,prop:8.2.3,prop:8.2.4,switching op sec 3}
  \begin{itemize}
    \item Ingredients: look at a $2$-separation and the corresponding subgraphs, use Theorem 3.2.25.b, use the switching operation of Section 3.2, use Lemma 8.2.6 and representations (8.2.3) and (8.2.4).
    \item Use the construction from the drawing, check that fundamental circuits match, conclude that $M$ is graphic. For planar graphs, the edge identification can be done in a planar way.
  \end{itemize}
\end{proof}


\subsection{Chapter 8.3}

\begin{proposition}[8.3.1] % todo: this defines 3-sum
  \label{prop:8.3.1}
  \uses{def:k_sep}
  Matrix $B$ with exact $k$-separation.
\end{proposition}

\begin{proposition}[8.3.2] % todo: this defines 3-sum
  \label{prop:8.3.2}
  \uses{prop:8.3.1,def:3_sum}
  Partition of $B$ displaying $k$-sum.
\end{proposition}

\begin{proposition}[8.3.9] % todo: this defines 3-sum
  \label{prop:8.3.9}
  \uses{prop:8.3.2,def:3_sum}
  The (well-chosen) matrix $\overline{B}$ representing the connecting minor $\overline{M}$ of a $3$-sum.
\end{proposition}

\begin{proposition}[8.3.10] % todo: this defines 3-sum
  \label{prop:8.3.10}
  \uses{prop:8.3.2,prop:8.3.9,def:3_sum}
  The matrix $B$ representing a $3$-sum (after reasoning).
\end{proposition}

\begin{proposition}[8.3.11]
  \label{prop:8.3.11}
  \uses{def:3_sum}
  Representation matrices $B^{1}$ and $B^{2}$ of the components $M_{1}$ and $M_{2}$ of a $3$-sum (after reasoning).
\end{proposition}

\begin{lemma}[8.3.12]
  \label{lem:8.3.12}
  \uses{def:k_conn,def:k_sep,def:binary_matroid,def:3_sum}
  Let $M$ be a $3$-connected binary matroid on a set $E$. Then any $3$-separation $(E_{1}, E_{2})$ of $M$ with $|E_{1}|, |E_{2}| \geq 4$ produces a $3$-sum, and vice versa.
\end{lemma}

\begin{proof}
  \uses{prop:8.3.1,prop:8.3.10,lem:2.3.14,prop:8.3.9}
  \begin{itemize}
    \item The converse easily follows from (8.3.10), which directly produces a desired $3$-separation.
    \item Take a $3$-separation. Since $M$ is $3$-connected, it must be exact. Consider the representation matrix (8.3.11). Reason about that matrix.
    \item Analyse shortest paths in a bipartite graph based on the matrix.
    \item Apply path shortening technique from Chapter 5 to reduce a shortest path by pivots to one with exactly two arcs.
    \item Reason about the corresponding entries and about the effects of the pivots on the matrix.
    \item Apply Lemma 2.3.14. Eventually get an instance of (8.3.10) with (8.3.9). Thus, $M$ is a $3$-sum.
  \end{itemize}
\end{proof}


\subsection{Chapter 8.5}

\begin{proposition}[8.5.3] % todo: this defines Delta sum
  \label{prop:8.5.3}
  \uses{prop:8.3.10,prop:8.3.11,prop:4.4.5}
  Matrix $B^{2 \Delta}$ for $M_{2 \Delta}$.
\end{proposition}
