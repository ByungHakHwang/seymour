\section{Matroid Definitions}

\begin{definition}[(finite) matroid]
  \label{def:code_finite_matroid}
  % \uses{}
  Let $E$ be a finite ground set. Let $\I \subseteq 2^{E}$ be a family of subsets satisfying:
  \begin{itemize}
    \item $\emptyset \in \I$ (non-empty)
    \item if $A \subseteq B \in \I$, then $A \in \I$ (down-closed)
    \item if $A ,B \in \I$ and $\left| A \right| < \left| B \right|$,
          then $A + b \in \I$ for some $b \in B \setminus A$ (exchange property)
  \end{itemize}
  Then the pair $M = \left(E, \I\right)$ is called a \emph{(finite) matroid}.
  % Note: Our definition is aligned with Mathlib
\end{definition}

\begin{definition}[binary matroid]
  \label{def:code_binary_matroid}
  \uses{def:code_finite_matroid}
  Let $B$ be a binary matrix, let $A = \left[ I \mid B \right]$, and let $E$ denote the column index set of $A$.
  Let $\I$ be all index subsets $Z \subseteq E$ such that the columns of $A$ indexed by $Z$ are independent over $\GFtwo$.
  Then $M = \left(E, \I\right)$ is called a \emph{binary matroid} and $B$ is called its \emph{(standard) representation matrix}.
  % Note: binary matroids may also be generated via the same process from arbirary rectangular binary matrices (not necessarily of the form $\left[ I \mid B \right]$)
  % Note: different binary matrices may generate the same binary matroid up to isomorphism (in both cases)
\end{definition}

\begin{definition}[regular matroid]
  \label{def:code_regular_matroid}
  \uses{def:code_binary_matroid,def:code_tu_matrix}
  Let $M$ be a binary matroid generated from a standard representation matrix $B$. Suppose $B$ has a TU signing, i.e., there exists a real matrix $A$ such that:
  \begin{itemize}
    \item $A$ is a signed version of $B$, i.e., $\left| A \right| = B$,
    \item $A$ is totally unimodular.
  \end{itemize}
  Then $M$ is called a \emph{regular matroid}.
\end{definition}
