\section{Matroid Definitions}

\begin{definition}[binary matroid]
  \label{BinaryMatroid}
  \lean{BinaryMatroid}
  \leanok
  Let $B$ be a binary matrix, let $A = \left[ I \mid B \right]$, and let $E$ denote the column index set of $A$.
  Let $\I$ be all index subsets $Z \subseteq E$ such that the columns of $A$ indexed by $Z$ are independent over $\GFtwo$.
  Then $M = \left(E, \I\right)$ is called a \emph{binary matroid} and $B$ is called its \emph{(standard) representation matrix}.
  % Note: binary matroids may also be generated via the same process from arbitrary rectangular binary matrices (not necessarily of the form $\left[ I \mid B \right]$)
  % Note: different binary matrices may generate the same binary matroid up to isomorphism (in both cases)
\end{definition}

\begin{definition}[regular matroid]
  \label{BinaryMatroid.IsRegular}
  \uses{BinaryMatroid,def:code_tu_matrix}
  \lean{BinaryMatroid.IsRegular}
  \leanok
  Let $M$ be a binary matroid generated from a standard representation matrix $B$. Suppose $B$ has a TU signing, i.e., there exists a real matrix $A$ such that:
  \begin{itemize}
    \item $A$ is a signed version of $B$, i.e., $\left| A \right| = B$,
    \item $A$ is totally unimodular.
  \end{itemize}
  Then $M$ is called a \emph{regular matroid}.
\end{definition}
