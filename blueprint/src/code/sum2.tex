\begin{definition}[$2$-sum of matrices]
  \label{Matrix_2sumComposition}
  \lean{Matrix_2sumComposition}
  % \uses{}
  \leanok
  Let $B$ be a matrix of the form
  \begin{tabular}{cccc}
                                 & $Y_{1}$                      & $Y_{2}$                      \\ \cline{2-3}
    \multicolumn{1}{c|}{$X_{1}$} & \multicolumn{1}{c|}{$A_{1}$} & \multicolumn{1}{c|}{    $0$} \\ \cline{2-3}
    \multicolumn{1}{c|}{$X_{2}$} & \multicolumn{1}{c|}{    $D$} & \multicolumn{1}{c|}{$A_{2}$} \\ \cline{2-3}
  \end{tabular}
  Let $B_{1}$ be a matrix of the form
  \begin{tabular}{cccc}
                                 & $Y_{1}$ \\ \cline{2-3}
    \multicolumn{1}{c|}{$X_{1}$} & \multicolumn{1}{c|}{$A_{1}$} \\ \cline{2-3}
    \multicolumn{1}{c|}{   Unit} & \multicolumn{1}{c|}{    $x$} \\ \cline{2-3}
  \end{tabular}
  Let $B_{2}$ be a matrix of the form
  \begin{tabular}{cccc}
                                 &                     Unit & $Y_{2}$ \\ \cline{2-3}
    \multicolumn{1}{c|}{$X_{2}$} & \multicolumn{1}{c|}{$y$} & \multicolumn{1}{c|}{$A_{2}$} \\ \cline{2-3}
  \end{tabular}
  Suppose that $\GFtwo \text{-rank } D = 1$, $x \neq 0$, $y \neq 0$, $D = y \cdot x$ (outer product).

  Then we say that $B_{1}$ and $B_{2}$ are the two \emph{components} of a \emph{$2$-sum decomposition} of $B$.

  Conversely, a \emph{$2$-sum composition} with \emph{components} $B_{1}$ and $B_{2}$ is the matrix $B$ above.

  The expression $B = B_{1} \oplus_{2} B_{2}$ means either process.
\end{definition}

\begin{definition}[matroid $2$-sum]
  \label{BinaryMatroid.Is2sumOf}
  \lean{BinaryMatroid.Is2sumOf}
  \uses{BinaryMatroid,Matrix_2sumComposition}
  \leanok
  Let $M$ be a binary matroid with a representation matrix $B$.
  Suppose $B$, $B_{1}$, and $B_{2}$ satisfy the assumptions of Definition~\ref{Matrix_2sumComposition}.
  Then the binary matroids $M_{1}$ and $M_{2}$ represented by $B_{1}$ and $B_{2}$, respectively, are the two \emph{components} of a \emph{$2$-sum decomposition} of $M$.

  Conversely, a \emph{$2$-sum composition} with \emph{components} $M_{1}$ and $M_{2}$ is the matroid $M$ defined by the corresponding representation matrix $B$.

  The expression $M = M_{1} \oplus_{2} M_{2}$ means either process.
\end{definition}

% todo: same as 1-separations and 1-sums
% \begin{lemma}[$2$-separations and $2$-sums of connected binary matroids]
%   \label{lem:code_2_sep_2_sum}
%   \uses{BinaryMatroid.Is2sumOf,def:code_k_sep,def:code_k_conn}
%   Let $M$ be a binary matroid that is $2$-separable.
%   Then $M$ can be decomposed as a $2$-sum with connected components given by the $2$-separation.

%   Conversely, any $2$-sum of two connected binary matroids is a connected binary matroid.
% \end{lemma}

% \begin{proof}[Proof sketch]
%   \uses{BinaryMatroid.Is2sumOf,def:code_k_sep,def:code_k_conn}
%   Check by definition. Connectedness of representation matrices is equivalent to connectedness of corresponding matroids.
% \end{proof}

\begin{lemma}[$2$-sum of TU matrices is a TU matrix]
  \label{Matrix_2sumComposition_TU}
  \lean{BinaryMatroid_2sum_isRegular}
  \uses{Matrix_2sumComposition,def:code_tu_matrix}
  \leanok
  Let $B_{1}$ and $B_{2}$ be TU matrices. Then $B = B_{1} \oplus_{2} B_{2}$ is a TU matrix.
\end{lemma}

\begin{proof}[Proof sketch]
  \uses{Matrix_2sumComposition,def:code_tu_matrix,lem:code_tu_add_ok_rows,cor:code_tu_add_ok_cols,lem:code_mvm_pivot,lem:code_mvm_props}
  % \leanok

  Let $B_{1}'$ and $B_{2}'$ be TU signings of $B_{1}$ and $B_{2}$, respectively.
  In particular, let $A_{1}'$, $x'$, $A_{2}'$, and $y'$ be the signed versions of $A_{1}$, $x$, $A_{2}$, and $y$, respectively.
  Let $B'$ be the signing of $B$ where the blocks of $A_{1}$ and $A_{2}$ are signed as $A_{1}'$ and $A_{2}'$, respectively, and the block of $D$ is signed as $D' = y' \cdot x'$ (outer product).

  Note that $\left[ A_{1}' / D' \right]$ is TU by Lemma~\ref{lem:code_tu_add_ok_rows}, as every row of $D'$ is either zero or a copy of $x'$.
  Similarly, $\left[D' \mid A_{2}' \right]$ is TU by Corollary~\ref{cor:code_tu_add_ok_cols}, as every column of $D'$ is either zero or a copy of $y'$.
  Additionally, $\left[ A_{1}' \mid 0 \right]$ is TU by Corollary~\ref{cor:code_tu_add_ok_cols}, and $\left[ 0 / A_{2}' \right]$ is TU by Lemma~\ref{lem:code_tu_add_ok_rows}.

  \todo[inline]{prove lemma below, separate into statement about TU matrices}

  \emph{Lemma:} Let $T$ be a square submatrix of $B'$. Then $\det T \in \{0, \pm 1\}$.

  \emph{Proof:} Induction on the size of $T$.
  \emph{Base:} If $T$ consists of only $1$ element, then this element is $0$ or $\pm 1$, so $\det T \in \{0, \pm 1\}$.
  \emph{Step:} Let $T$ have size $t$ and suppose all square submatrices of $B'$ of size $\leq t - 1$ are TU.
  \begin{itemize}
    \item Suppose $T$ contains no rows of $X_{1}$. Then $T$ is a submatrix of $\left[D' \mid A_{2}' \right]$, so $\det T \in \{0, \pm 1\}$.
    \item Suppose $T$ contains no rows of $X_{2}$. Then $T$ is a submatrix of $\left[A_{1}' \mid 0 \right]$, so $\det T \in \{0, \pm 1\}$.
    \item Suppose $T$ contains no columns of $Y_{1}$. Then $T$ is a submatrix of $\left[0 / A_{2}' \right]$, so $\det T \in \{0, \pm 1\}$.
    \item Suppose $T$ contains no columns of $Y_{2}$. Then $T$ is a submatrix of $\left[ A_{1}' / D' \right]$, so $\det T \in \{0, \pm 1\}$.
    \item Remaining case: $T$ contains rows of $X_{1}$ and $X_{2}$ and columns of $Y_{1}$ and $Y_{2}$.
    \item If $T$ is $2 \times 2$, then $T$ is TU. Indeed, all proper submatrices of $T$ are of size $\leq 1$, which are $\{0, \pm 1\}$ entries of $A'$, and $T$ contains a zero entry (in the row of $X_{2}$ and column of $Y_{2}$), so it cannot be a minimal violation matrix by Lemma~\ref{lem:code_mvm_props}. Thus, assume $T$ has size $\geq 3$.
    \item .
    \todo[inline]{complete proof, see last paragraph of Lemma 11.2.1 in Truemper} % lem:code_pivot_tu
  \end{itemize}
\end{proof}

\begin{theorem}[$2$-sum of regular matroids is a regular matroid]
  \label{BinaryMatroid.Is2sumOf.isRegular}
  \lean{BinaryMatroid.Is2sumOf.isRegular}
  \uses{BinaryMatroid.Is2sumOf,BinaryMatroid.IsRegular}
  \leanok
  Let $M_{1}$ and $M_{2}$ be regular matroids. Then $M = M_{1} \oplus_{2} M_{2}$ is a regular matroid.
\end{theorem}

\begin{proof}[Proof sketch]
  \uses{BinaryMatroid.Is2sumOf,BinaryMatroid.IsRegular,Matrix_2sumComposition_TU,BinaryMatroid_toMatroid_isRegular_iff}
  \leanok
  Let $B$, $B_{1}$, and $B_{2}$ be the representation matrices of $M$, $M_{1}$, and $M_{2}$, respectively.
  Apply Lemma~\ref{Matrix_2sumComposition_TU}.
\end{proof}

\begin{lemma}[left summand of regular $2$-sum is regular]
  \label{BinaryMatroid.Is2sumOf.isRegular_left}
  \lean{BinaryMatroid.Is2sumOf.isRegular_left}
  \uses{BinaryMatroid.Is2sumOf,BinaryMatroid.IsRegular}
  \leanok
  \todo[inline]{add}
\end{lemma}

\begin{lemma}[right summand of regular $2$-sum is regular]
  \label{BinaryMatroid.Is2sumOf.isRegular_right}
  \lean{BinaryMatroid.Is2sumOf.isRegular_right}
  \uses{BinaryMatroid.Is2sumOf,BinaryMatroid.IsRegular}
  \leanok
  \todo[inline]{add}
\end{lemma}
