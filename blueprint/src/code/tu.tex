\section{TU Matrices}

\begin{definition}[TU matrix]
  \label{def:code_tu_matrix}
  \lean{Matrix.TU}
  \leanok
  A real matrix is \emph{totally unimodular (TU)} if its every subdeterminant (i.e., determinant of every square submatrix) is $0$ or $\pm 1$.
\end{definition}

\begin{lemma}[entries of a TU matrix]
  \label{lem:code_tu_entries}
  \uses{def:code_tu_matrix}
  \lean{Matrix.TU.apply}
  \leanok
  If $A$ is TU, then every entry of $A$ is $0$ or $\pm 1$.
\end{lemma}

\begin{proof}[Proof sketch]
  \uses{def:code_tu_matrix}
  \leanok
  Every entry is a square submatrix of size $1$, and therefore has determinant (and value) $0$ or $\pm 1$.
\end{proof}

\begin{lemma}[any submatrix of a TU matrix is TU]
  \label{lem:code_submatrix_of_tu}
  \uses{def:code_tu_matrix}
  \lean{Matrix.TU.submatrix}
  \leanok
  Let $A$ be a real matrix that is TU and let $B$ be a submatrix of $A$. Then $B$ is TU.
\end{lemma}

\begin{proof}[Proof sketch]
  \uses{def:code_tu_matrix}
  \leanok
  Any square submatrix of $B$ is a submatrix of $A$, so its determinant is $0$ or $\pm 1$. Thus, $B$ is TU.
\end{proof}

\begin{lemma}[transpose of TU is TU]
  \label{lem:code_tu_transpose}
  \uses{def:code_tu_matrix}
  \lean{Matrix.TU.transpose}
  \leanok
  Let $A$ be a TU matrix. Then $A^{T}$ is TU.
\end{lemma}

\begin{proof}[Proof sketch]
  \uses{def:code_tu_matrix}
  \leanok
  A submatrix $T$ of $A^{T}$ is a transpose of a submatrix of $A$, so $\det T \in \{0, \pm 1\}$.
\end{proof}

\begin{lemma}[appending zero vector to TU]
  \label{lem:code_tu_add_zero_row}
  \uses{def:code_tu_matrix}
  \lean{Matrix.TU_adjoin_row0s_iff}
  \leanok
  Let $A$ be a matrix. Let $a$ be a zero row. Then $C = \left[ A / a \right]$ is TU exactly when $A$ is.
\end{lemma}

\begin{proof}[Proof sketch]
  \uses{def:code_tu_matrix,lem:code_submatrix_of_tu}
  \leanok
  Let $T$ be a square submatrix of $C$, and suppose $A$ is TU. If $T$ contains a zero row, then $\det T = 0$. Otherwise $T$ is a submatrix of $A$, so $\det T \in \{0, \pm 1\}$.
  For the other direction, because $A$ is a submatrix of $C$, we can apply lemma \ref{lem:code_submatrix_of_tu}.
\end{proof}

\begin{lemma}[appending unit vector to TU]
  \label{lem:code_tu_add_unit_row}
  \uses{def:code_tu_matrix}
  \lean{Matrix.TU_adjoin_rowUnit_iff}
  \leanok
  Let $A$ be a matrix. Let $a$ be a unit row. Then $C = \left[ A / a \right]$ is TU exactly when $A$ is.
\end{lemma}

\begin{proof}[Proof sketch]
  \uses{def:code_tu_matrix,lem:code_submatrix_of_tu}
  Let $T$ be a square submatrix of $C$, and suppose $A$ is TU.
  If $T$ contains the $\pm 1$ entry of the unit row,
  then $\det T$ equals the determinant of some submatrix of $A$ times $\pm 1$,
  so $\det T \in \{0, \pm 1\}$.
  If $T$ contains some entries of the unit row except the $\pm 1$, then $\det T = 0$.
  Otherwise $T$ is a submatrix of $A$, so $\det T \in \{0, \pm 1\}$.
  For the other direction, simply note that $A$ is a submatrix of $C$, and use lemma \ref{lem:code_submatrix_of_tu}.
\end{proof}

\begin{lemma}[TUness with adjoint identity matrix]
  \label{lem:code_tu_adjoin_id}
  \uses{def:code_tu_matrix}
  \lean{Matrix.TU_adjoin_id_below_iff,Matrix.TU_adjoin_id_above_iff,Matrix.TU_adjoin_id_left_iff,Matrix.TU_adjoin_id_right_iff}
  \leanok
  $A$ is TU iff every basis matrix of $\left[ I \mid A \right]$ has determinant $\pm 1$.
\end{lemma}

\begin{proof}[Proof sketch]
  \uses{def:code_tu_matrix}
  Gaussian elimination. Basis submatrix: its columns form a basis of all columns, its rows form a basis of all rows.
\end{proof}

\begin{lemma}[block-diagonal matrix with TU blocks is TU]
  \label{lem:code_diagonal_with_tu_blocks}
  \uses{def:code_tu_matrix}
  \lean{Matrix.fromBlocks_TU}
  \leanok
  Let $A$ be a matrix of the form
  \begin{tabular}{cc}
    \cline{1-2}
    \multicolumn{1}{|c|}{$A_{1}$} & \multicolumn{1}{c|}{    $0$} \\ \cline{1-2}
    \multicolumn{1}{|c|}{    $0$} & \multicolumn{1}{c|}{$A_{2}$} \\ \cline{1-2}
  \end{tabular}
  where $A_{1}$ and $A_{2}$ are both TU. Then $A$ is also TU.
\end{lemma}

\begin{proof}[Proof sketch]
  \uses{def:code_tu_matrix}
  Any square submatrix $T$ of $A$ has the form
  \begin{tabular}{cc}
    \cline{1-2}
    \multicolumn{1}{|c|}{$T_{1}$} & \multicolumn{1}{c|}{    $0$} \\ \cline{1-2}
    \multicolumn{1}{|c|}{    $0$} & \multicolumn{1}{c|}{$T_{2}$} \\ \cline{1-2}
  \end{tabular}
  where $T_{1}$ and $T_{2}$ are submatrices of $A_{1}$ and $A_{2}$, respectively.
  \begin{itemize}
    \item If $T_{1}$ is square, then $T_{2}$ is also square, and $\det T = \det T_{1} \cdot \det T_{2} \in \{0, \pm 1\}$.
    \item If $T_{1}$ has more rows than columns, then the rows of $T$ containing $T_{1}$ are linearly dependent, so $\det T = 0$.
    \item Similar if $T_{1}$ has more columns than rows.
  \end{itemize}
\end{proof}

\begin{lemma}[appending parallel element to TU]
  \label{lem:code_tu_add_copy_row}
  \uses{def:code_tu_matrix}
  % \lean{} % todo: add reference to corresponding lemma in lean
  Let $A$ be a TU matrix. Let $a$ be some row of $A$. Then $C = \left[ A / a \right]$ is TU.
\end{lemma}

\begin{proof}[Proof sketch]
  \uses{def:code_tu_matrix}
  Let $T$ be a square submatrix of $C$. If $T$ contains the same row twice, then the rows are \GFtwo-dependent, so $\det T = 0$. Otherwise $T$ is a submatrix of $A$, so $\det T \in \{0, \pm 1\}$.
\end{proof}

\begin{lemma}[appending rows to TU]
  \label{lem:code_tu_add_ok_rows}
  \uses{def:code_tu_matrix}
  % \lean{} % todo: add reference to corresponding lemma in lean
  Let $A$ be a TU matrix. Let $B$ be a matrix whose every row is a row of $A$, a zero row, or a unit row. Then $C = \left[ A / B \right]$ is TU.
\end{lemma}

\begin{proof}[Proof sketch]
  \uses{def:code_tu_matrix,lem:code_tu_add_zero_row,lem:code_tu_add_unit_row,lem:code_tu_add_copy_row}
  Either repeatedly apply Lemmas~\ref{lem:code_tu_add_zero_row}, \ref{lem:code_tu_add_unit_row}, and \ref{lem:code_tu_add_copy_row}
  or perform a similar case analysis directly.
\end{proof}

\begin{corollary}[appending columns to TU]
  \label{cor:code_tu_add_ok_cols}
  \uses{def:code_tu_matrix,lem:code_tu_add_zero_row,lem:code_tu_add_unit_row,lem:code_tu_add_copy_row}
  % \lean{} % todo: add reference to corresponding lemma in lean
  Let $A$ be a TU matrix. Let $B$ be a matrix whose every column is a column of $A$, a zero column, or a unit column. Then $C = \left[ A \mid B \right]$ is TU.
\end{corollary}

\begin{proof}[Proof sketch]
  \uses{def:code_tu_matrix,lem:code_tu_add_zero_row,lem:code_tu_add_unit_row,lem:code_tu_add_copy_row,lem:code_tu_transpose}
  $C^{T}$ is TU by Lemma~\ref{lem:code_tu_add_ok_rows} and construction, so $C$ is TU by Lemma~\ref{lem:code_tu_transpose}.
\end{proof}

\begin{definition}[$\mathcal{F}$-pivot]
  \label{def:code_pivot}
  % \lean{} % todo: add reference to corresponding lemma in lean
  Let $A$ be a matrix over a field $\mathcal{F}$ with row index set $X$ and column index set $Y$.
  Let $A_{xy}$ be a nonzero element.
  The result of a \emph{$\mathcal{F}$-pivot} of $A$ on the \emph{pivot element} $A_{xy}$
  is the matrix $A'$ over $\mathcal{F}$ with row index set $X'$ and column index set $Y'$ defined as follows.
  \begin{itemize}
    \item For every $u \in X - x$ and $w \in Y - y$, let $A_{uw}' = A_{uw} + (A_{uy} \cdot A_{xw}) / (-A_{xy})$.
    \item Let $A_{xy}' = -A_{xy}$, $X' = X - x + y$, and $Y' = Y - y + x$.
  \end{itemize}
\end{definition}

\begin{lemma}[pivoting preserves TUness]
  \label{lem:code_pivot_tu}
  \uses{def:code_tu_matrix,def:code_pivot}
  % \lean{} % todo: add reference to corresponding lemma in lean
  Let $A$ be a TU matrix and let $A_{xy}$ be a nonzero element.
  Let $A'$ be the matrix obtained by performing a real pivot in $A$ on $A_{xy}$.
  Then $A'$ is TU.
\end{lemma}

\begin{proof}[Proof sketch]
  \uses{def:code_tu_matrix,def:code_pivot,lem:code_tu_adjoin_id}
  \begin{itemize}
    \item By Lemma~\ref{lem:code_tu_adjoin_id} $A$ is TU iff every basis matrix of $\left[ I \mid A \right]$ has determinant $\pm 1$. The same holds for $A'$ and $\left[ I \mid A' \right]$.
    \item Determinants of the basis matrices are preserved under elementary row operations in $\left[ I \mid A \right]$ corresponding to the pivot in $A$, under scaling by $\pm 1$ factors, and under column exchange, all of which together convert $\left[ I \mid A \right]$ to $\left[ I \mid A' \right]$.
  \end{itemize}
\end{proof}

\begin{lemma}[pivoting preserves TUness]
  \label{lem:code_reverse_pivot_tu}
  \uses{def:code_tu_matrix,def:code_pivot}
  % \lean{} % todo: add reference to corresponding lemma in lean
  Let $A$ be a matrix and let $A_{xy}$ be a nonzero element.
  Let $A'$ be the matrix obtained by performing a real pivot in $A$ on $A_{xy}$.
  If $A'$ is TU, then $A$ is TU.
\end{lemma}

\begin{proof}[Proof sketch]
  \uses{def:code_tu_matrix,def:code_pivot,lem:code_pivot_tu}
  Reverse the row operations, scaling, and column exchange in the proof of Lemma~\ref{lem:code_pivot_tu}.
\end{proof}


\subsection{Minimal Violation Matrices}

\begin{definition}[minimal violation matrix]
  \label{def:code_mvm}
  \uses{def:code_tu_matrix}
  % \lean{} % todo: add reference to corresponding lemma in lean
  Let $A$ be a real $\{0, \pm 1\}$ matrix that is not TU but all of whose proper submatrices are TU.
  Then $A$ is called a \emph{minimal violation matrix of total unimodularity (minimal violation matrix)}.
\end{definition}

\begin{lemma}[simple properties of MVMs]
  \label{lem:code_mvm_props}
  \uses{def:code_mvm}
  % \lean{} % todo: add reference to corresponding lemma in lean
  Let $A$ be a minimal violation matrix.
  \begin{itemize}
    \item $A$ is square.
    \item $\det A \notin \{0, \pm 1\}$.
    \item If $A$ is $2 \times 2$, then $A$ does not contain a $0$.
  \end{itemize}
\end{lemma}

\begin{proof}[Proof sketch]
  \uses{def:code_mvm}
  \begin{itemize}
    \item If $A$ is not square, then since all its proper submatrices are TU, $A$ is TU, contradiction.
    \item If $\det A \in \{0, \pm 1\}$, then all subdeterminants of $A$ are $0$ or $\pm 1$, so $A$ is TU, contradiction.
    \item If $A$ is $2 \times 2$ and it contains a $0$, then $\det A \in \{\pm 1\}$, which contradicts the previous item.
  \end{itemize}
\end{proof}

\begin{lemma}[pivoting in MVMs]
  \label{lem:code_mvm_pivot}
  \uses{def:code_mvm,def:code_pivot}
  % \lean{} % todo: add reference to corresponding lemma in lean
  Let $A$ be a minimal violation matrix. Suppose $A$ has $\geq 3$ rows.
  Suppose we perform a real pivot in $A$, then delete the pivot row and column.
  Then the resulting matrix $A'$ is also a minimal violation matrix.
\end{lemma}

\begin{proof}[Proof sketch]
  \uses{def:code_mvm,lem:code_diagonal_with_tu_blocks,lem:code_reverse_pivot_tu,lem:code_pivot_tu,lem:code_submatrix_of_tu}
  \begin{itemize}
    \item Let $A''$ denote matrix $A$ after the pivot, but before the pivot row and column are deleted.
    \item Since $A$ is not TU, Lemma~\ref{lem:code_reverse_pivot_tu} implies that $A''$ is not TU. Thus $A'$ is not TU by Lemma~\ref{lem:code_diagonal_with_tu_blocks}.
    \item Let $T'$ be a proper square submatrix of $A'$. Let $T''$ be the submatrix of $A''$ consisting of $T'$ plus the pivot row and the pivot column, and let $T$ be the corresponding submatrix of $A$ (defined by the same row and column indices as $T''$).
    \item $T$ is TU as a proper submatrix of $A$. Then Lemma~\ref{lem:code_pivot_tu} implies that $T''$ is TU. Thus $T'$ is TU by Lemma~\ref{lem:code_submatrix_of_tu}.
  \end{itemize}
\end{proof}
