\section{Chapter 5 from Truemper}

\begin{lemma}[5.2.4]
  \label{lem:5.2.4}
  \uses{def:k_conn,def:minor,def:1_elem_ext}
  Let $N$ be a connected minor of a connected binary matroid $M$. Let $z \in M \setminus N$. Then $M$ has a connected minor $N'$ that is a $1$-element extension of $N$ by $z$.
\end{lemma}

\begin{proof}[Proof sketch]
  \uses{lem:3.3.12}
  \begin{itemize}
    \item By Lemma 3.3.12, $M$ has a representation matrix that displays $N$ via a submatrix.
    \item Case distinction between $z$ being represented by a nonzero or a zero vector.
    \item Nonzero case: immediately get submatrix representing $N'$.
    \item Zero case: take a shortest path in the matrix, perform pivots, in one subcase use duality.
  \end{itemize}
\end{proof}

\begin{proposition}[5.2.8]
  \label{prop:5.2.8}
  \uses{def:wheel}
  Representation matrices for small wheels (from $M(W_{1})$ to $M(W_{4})$).
\end{proposition}

\begin{proposition}[5.2.9]
  \label{prop:5.2.9}
  \uses{def:wheel}
  Representation matrix for $M(W_{n})$, $n \geq 3$.
\end{proposition}

\begin{lemma}[5.2.10]
  \label{lem:5.2.10}
  \uses{def:binary_matroid,def:minor,def:wheel}
  Let $M$ be a binary matroid with a binary representation matrix $B$. Suppose the graph $BG(B)$ contains at least one cycle. Then $M$ has an $M(W_{2})$ minor.
\end{lemma}

\begin{proof}[Proof sketch]
  \uses{prop:5.2.8,prop:5.2.9}
  \begin{itemize}
    \item $BG(B)$ is bipartite and has at least one cycle, so there is a cycle $C$ without chords with at least $4$ edges.
    \item Up to indices, the submatrix corresponding to $C$ is either the matrix for $M(W_{2})$ from (5.2.8) or the matrix for some $M(W_{k})$, $k \geq 3$ from (5.2.9).
    \item In the latter case, use path shortening pivots on $1$s to convert the submatrix to the former case.
  \end{itemize}
\end{proof}

\begin{lemma}[5.2.11]
  \label{lem:5.2.11}
  \uses{def:binary_matroid,def:k_conn,def:k_sep,def:M_W_3,def:minor}
  Let $M$ be a connected binary matroid with at least $4$ elements. Then $M$ has a $2$-separation or an $M(W_3)$ minor.
\end{lemma}

\begin{proof}[Proof sketch]
  \uses{lem:5.2.10}
  Use Lemma 5.2.10 and apply path shortening technique.
\end{proof}

\begin{corollary}[5.2.15]
  \label{cor:5.2.15}
  \uses{def:k_conn,def:binary_matroid,def:M_W_3,def:minor}
  Every $3$-connected binary matroid $M$ with at least $6$ elements has an $M(W_3)$ minor.
\end{corollary}

\begin{proof}[Proof sketch]
  \uses{lem:5.2.11}
  By Lemma 5.2.11, $M$ has a $2$-separation or an $M(W_{3})$ minor. $M$ is $3$-connected, so the former case is impossible.
\end{proof}
