\section{Chapter 6 from Truemper}

\begin{proposition}[6.3.12]
  \label{prop:6.3.12}
  \todo[inline]{add}
\end{proposition}

\begin{proposition}[6.3.21]
  \label{prop:6.3.21}
  \todo[inline]{add}
\end{proposition}

\begin{proposition}[6.3.22]
  \label{prop:6.3.22}
  \todo[inline]{add}
\end{proposition}

\begin{proposition}[6.3.23]
  \label{prop:6.3.23}
  \todo[inline]{add}
\end{proposition}

\begin{corollary}[6.3.24]
  \label{cor:6.3.24}
  \uses{def:binary_matroid,def:isomorphism,def:minor,def:1_elem_ext,def:2_elem_ext,prop:6.3.12,prop:6.3.21,prop:6.3.22,prop:6.3.23,def:k_sep}
  Let $\M$ be a class of binary matroids closed under isomorphism and under taking minors. Suppose $N$ given by $B^{N}$ of (6.3.12) is in $\M$, but the $1$- and $2$-element extensions of $N$ given by (6.3.21), (6.3.22), (6.3.23), and by the accompanying conditions are not in $\M$. Assume matroid $M \in \M$ has an $N$ minor.
  Then any $k$-separation of any such minor that corresponds to $(X_{1} \cup Y_{1}, X_{2} \cup Y_{2})$ of $N$ under one of the isomorphisms induces a $k$-separation of $M$.
\end{corollary}

\begin{theorem}[6.4.1]
  \label{thm:6.4.1}
  \todo[inline]{add}
\end{theorem}
