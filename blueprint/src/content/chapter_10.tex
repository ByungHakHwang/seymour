\section{Chapter 10 from Truemper}

\begin{proposition}[10.2.4]
  \label{prop:10.2.4}
  \todo[inline]{expand}
  Derivation of a graph with $T$ nodes for $F_{7}$.
\end{proposition}

\begin{proposition}[10.2.6]
  \label{prop:10.2.6}
  \todo[inline]{expand}
  Derivation of a graph with $T$ nodes for $M(K_{3,3})^{*}$.
\end{proposition}

\begin{proposition}[10.2.8]
  \label{prop:10.2.8}
  \todo[inline]{expand}
  Derivation of a graph with $T$ nodes for $R_{10}$.
\end{proposition}

\begin{proposition}[10.2.9]
  \label{prop:10.2.9}
  \todo[inline]{expand}
  Derivation of a graph with $T$ nodes for $R_{12}$.
\end{proposition}

\begin{theorem}[10.2.11 only if]
  \label{thm:10.2.11.only_if}
  \uses{def:regular_matroid,def:planar_matroid,def:M_K_5,def:M_K_5_dual,def:M_K_3_3,def:M_K_3_3_dual,def:minor}
  If a regular matroid is planar, then it has no $M(K_{5})$, $M(K_{5})^{*}$, $M(K_{3,3})$, or $M(K_{3,3})^{*}$ minors.
\end{theorem}

\begin{proof}[Proof sketch]
  \begin{itemize}
    \item Planarity is preserved under taking minors.
    \item The listed matroids are not planar.
  \end{itemize}
\end{proof}

\begin{theorem}[10.2.11 if]
  \label{thm:10.2.11.if}
  \uses{def:regular_matroid,def:planar_matroid,def:M_K_5,def:M_K_5_dual,def:M_K_3_3,def:M_K_3_3_dual,def:minor}
  If a regular matroid has no $M(K_{5})$, $M(K_{5})^{*}$, $M(K_{3,3})$, or $M(K_{3,3})^{*}$ minors, then it is planar.
\end{theorem}

\begin{proof}[Proof sketch]
  \uses{thm:7.4.1,lem:8.2.2,lem:8.2.6,lem:8.2.7,census sec 3.3,thm:7.3.3,prop:10.2.4,prop:10.2.6,thm:Menger}
  \begin{itemize}
    \item Let $M$ be minimally nonplanar with respect to taking minors, i.e., regular nonplanar, but with all proper minors planar.
    \item Goal: show that $M$ is isomorphic to one of the listed matroids.
    \item By Theorem 7.4.1, $M$ is not graphic or cographic.
    \item By Lemmas 8.2.2, 8.2.6, and 8.2.7, if $M$ has a $1$- or $2$-separation, then $M$ is a $1$- or $2$-sum. But then the components of the sum are planar, so $M$ is also planar. Therefore, $M$ is $3$-connected.
    \item By the census of Section 3.3, every $3$-connected $\leq 8$-element matroid is planar, so $|M| \geq 9$.
    \item By the binary matroid version of the wheel Theorem 7.3.3, there exists an element $z$ such that $M \\ z$ or $M / z$ is $3$-connected. Dualizing does not afect the assumptions, so we may assume that $M \\ z$ is $3$-connected.
    \item Let $G$ be a planar graph representing $M \\ z$. Extend $G$ to a representation of $M$ as follows:
    \begin{itemize}
      \item If $G$ is a wheel, invoke (10.2.6) or (10.2.4). The latter contracdicts regularity of $M$, the former shows what we need.
      \item If $G$ is not a wheel, use Theorem 7.3.3 and Menger's theorem. Use a path argument and edge contraction to reduce to (10.2.6) and conclude the proof.
    \end{itemize}
  \end{itemize}
\end{proof}

\begin{lemma}[10.3.1]
  \label{lem:10.3.1}
  \uses{def:M_K_5,def:splitter,def:regular_matroid,def:M_K_3_3,def:minor}
  $M(K_5)$ is a splitter of the regular matroids with no $M(K_{3,3})$ minors.
\end{lemma}

\begin{proof}
  \uses{thm:7.2.1.a,def:k_conn,def:1_elem_ext}
  \begin{itemize}
    \item By Theorem 7.2.1.a, we only need to show that every $3$-connected regular $1$-element extension of $M(K_5)$ has an $M(K_{3,3})$ minor.
    \item Then case analysis. (The book sketches one way of checking.)
  \end{itemize}
\end{proof}

\begin{lemma}[10.3.6]
  \label{lem:10.3.6}
  \uses{def:k_conn,def:1_elem_ext,def:M_K_3_3,def:regular_matroid,def:binary_matroid}
  Every $3$-connected binary $1$-element expansion of $M(K_{3,3})$ is nonregular.
\end{lemma}

\begin{proof}[Proof sketch]
  By case analysis via graphs plus $T$ sets.
\end{proof}

\begin{theorem}[10.3.11]
  \label{thm:10.3.11}
  \uses{def:k_conn,def:regular_matroid,def:M_K_3_3,def:minor,def:graphic_matroid,def:cographic_matroid,def:isomorphism,def:R10,def:R12}
  Let $M$ be a $3$-connected regular matroid with an $M(K_{3,3})$ minor. Assume that $M$ is not graphic and not cographic, but that each proper minor of $M$ is graphic or cographic.
  Then $M$ is isomorphic to $R_{10}$ or $R_{12}$.
\end{theorem}

\begin{proof}
  \uses{lem:10.3.6,thm:7.3.4,thm:Menger}
  This proof is extremely long and technical. It involves case distinctions and graph constructions.
\end{proof}

\begin{theorem}[10.4.1 only if]
  \label{thm:10.4.1.only_if}
  \uses{def:k_conn,def:regular_matroid,def:graphic_matroid,def:cographic_matroid,def:R10,def:R12,def:minor}
  If $3$-connected regular matroid is graphic or cographic, then it has no $R_{10}$ or $R_{12}$ minors.
\end{theorem}

\begin{proof}[Proof sketch]
  \uses{prop:10.2.8,prop:10.2.9}
  % todo: alternative breakdown:
  % in proof sketch: \uses{lem:R10_nongraphic,lem:R10_noncographic,lem:R12_nongraphic,lem:R12_noncographic}
  % in \label{lem:R10_nongraphic}: \uses{prop:10.2.8}
  % in \label{lem:R12_nongraphic}: \uses{prop:10.2.9}
  % in \label{lem:R10_noncographic}: \uses{lem:R10_nongraphic,lem:R10_selfdual}
  % in \label{lem:R12_noncographic}: \uses{lem:R12_nongraphic,lem:R12_ introduces  andselfdual}
  Representations (10.2.8) and (10.2.9) for $R_{10}$ and $R_{12}$ show that these are nongraphic and isomorphic to their duals, hence also noncographic, so we are done.
\end{proof}

\begin{theorem}[10.4.1 if]
  \label{thm:10.4.1.if}
  \uses{def:k_conn,def:regular_matroid,def:graphic_matroid,def:cographic_matroid,def:R10,def:R12,def:minor}
  If a $3$-connected regular matroid has no $R_{10}$ or $R_{12}$ minors, then it is graphic or cographic.
\end{theorem}

\begin{proof}[Proof sketch]
  \uses{thm:10.2.11.if,lem:10.3.1,thm:10.3.11}
  \begin{itemize}
    \item Let $M$ be $3$-connected, regular, nongraphic, and noncographic matroid.
    \item Thus $M$ is not planar, so by Theorem 10.2.11 it has a minor isomorphic to $M(K_{5})$, $M(K_{5})^{*}$, $M(K_{3,3})$, or $M(K_{3,3})^{*}$.
    \item By Lemma 10.3.1, $M(K_{5})$ is a splitter for the regular matroids with no $M(K_{3,3})$ minors.
    \item These results imply that $M$ has a minor isomorphic to $M(K_{3,3})$, or $M(K_{3,3})^{*}$, or $M$ is isomorphic to $M(K_{5})$ or $M(K_{5})^{*}$.
    \item The latter is a contradiction, so $M$ or $M^{*}$ has an $M(K_{3,3})$ minor.
    \item Theorem 10.3.11 implies that $M$ or $M^{*}$ has $R_{10}$ or $R_{12}$ as a minor.
    \item Since $R_{10}$ and $R_{12}$ are self-dual, $M$ has $R_{10}$ or $R_{12}$ as a minor.
  \end{itemize}
\end{proof}

Note: Truemper's proof of \ref{thm:10.4.1.if} and \ref{thm:10.4.1.only_if} relies on representing matroids via graphs plus $T$ sets. An alternative proof, which utilizes the notion of graph signings, can be found in \href{https://www.math.uwaterloo.ca/~jfgeelen/Publications/regular.pdf}{J. Geelen, B. Gerards - Regular matroid decomposition via signed graphs}. Although the proof appears shorter than Truemper's, it heavily relies certain relatively advanced graph-theoretic results.

Bonus: Whitney's characterization of planar graphs (Corollary 10.2.13).
