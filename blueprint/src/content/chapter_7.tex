\section{Chapter 7 from Truemper}

\begin{definition}[splitter]
  \label{def:splitter}
  Let $\M$ be a class of binary matroids closed under isomorphism and under taking minors. Let $N$ be a $3$-connected minor of $\M$ on at least $6$ elements.
  If every $M \in \M$ with a proper $N$ minor has a $2$-separation, then $N$ is called a splitter of $\M$.
\end{definition}

\begin{theorem}[7.2.1.a splitter for nonwheels]
  \label{thm:7.2.1.a}
  \uses{def:splitter,def:wheel}
  Let $\M$ be a class of binary matroids closed under isomorphism and under taking minors. Let $N$ be a $3$-connected minor of $\M$ on at least $6$ elements.
  If $N$ is not a wheel, then $N$ is a splitter of $\M$ iff $\M$ does not contain a $3$-connected $1$-element extension of $N$.
\end{theorem}

\begin{proof}[Proof sketch]
  \uses{thm:6.4.1,def:splitter,def:k_conn,def:1_elem_ext,def:minor,def:k_sep}
  \begin{itemize}
    \item If $N$ is a splitter of $\M$, then clearly $\M$ does not contain a $3$-connected $1$-element extension of $N$.
    \item Prove the converse by contradiction. To this end, suppose that $\M$ does not contain a $3$-connected $1$-element extension of $N$ and that $N$ is not a splitter of $\M$.
    \item Thus, $\M$ contains a $3$-connected matroid $M$ with a proper $N$ minor and no $2$-separation.
    \item Since $\M$ is closed under isomorphism, we may assume $N$ itself to be that $N$ minor.
    \item By Theorem 6.4.1 (applied to $M$ and $N$), $M$ has a $3$-connected minor $N'$ that is a $3$-connected $1$- or $2$-element extension of an $N$ minor.
    \item The $1$-extension case has been ruled out.
    \item In the $2$-element extension case, $N'$ is derived from the $N$ minor by one addition and one expansion. Again, since $\M$ is closed under isomorphism and minor taking, we may take $N$ itself to be that $N$ minor. Thus, $N'$ is derived from $N$ by one addition and one expansion.
    \item Let $C$ be a binary matrix representing $N'$ and displaying $N$. By investigating the structure of $C$, one can show that $N'$ contains a $3$-connected $1$-element extension of an $N$ minor, which has been ruled out.
  \end{itemize}
\end{proof}

\begin{theorem}[7.2.1.b splitter for wheels]
  \label{thm:7.2.1.b}
  \uses{def:splitter,def:wheel}
  Let $\M$ be a class of binary matroids closed under isomorphism and under taking minors. Let $N$ be a $3$-connected minor of $\M$ on at least $6$ elements.
  If $N$ is a wheel, then $N$ is a splitter of $\M$ iff $\M$ does not contain a $3$-connected $1$-element extension of $N$ and does not contain the next larger wheel.
\end{theorem}

\begin{proof}[Proof sketch]
  \uses{thm:6.4.1,def:splitter,def:k_conn,def:1_elem_ext,def:minor,def:k_sep}
  Similar to proof of Theorem 7.2.1.a. The analysis of the matrix $C$ can be done in one go for both cases.
\end{proof}

\begin{corollary}[7.2.10.a]
  \label{cor:7.2.10.a}
  Theorem 7.2.1.a specialized to graphs.
\end{corollary}

\begin{proof}[Proof sketch]
  \uses{thm:7.2.1.a}
  Consider the corresponding graphic matroids, apply splitter theorem, extensions in graphic matroids correspond to extensions in graphs.
\end{proof}

\begin{corollary}[7.2.10.b]
  \label{cor:7.2.10.b}
  Theorem 7.2.1.b specialized to graphs.
\end{corollary}

\begin{proof}[Proof sketch]
  \uses{thm:7.2.1.b}
  Consider the corresponding graphic matroids, apply splitter theorem, extensions in graphic matroids correspond to extensions in graphs.
\end{proof}

\begin{theorem}[7.2.11.a]
  \label{thm:7.2.11.a}
  \uses{def:M_K_5,def:M_K_3_3,def:splitter,def:minor,def:graphic_matroid}
  $K_{5}$ is a splitter of the graphs without $K_{3,3}$ minors.
\end{theorem}

\begin{proof}[Proof sketch]
  \uses{cor:7.2.10.a,def:k_conn,def:1_elem_ext}
  Up to isomorphism, there is just one $3$-connected $1$-edge extension of $K_5$. To obtain it, one partitions one vertex of $K_5$ into two vertices of degree $2$ and connects the two vertices by a new edge. The resulting graph has a $K_{3,3}$ minor. Thus, the theorem follows fromCorollary 7.2.10.a.
\end{proof}

\begin{theorem}[7.2.11.b]
  \label{thm:7.2.11.b}
  \uses{def:M_W_3,def:M_W_4,def:splitter,def:minor,def:graphic_matroid}
  $W_{3}$ is a splitter of the graphs without $W_{4}$ minors.
\end{theorem}

\begin{proof}[Proof sketch]
  \uses{cor:7.2.10.b,def:k_conn,def:1_elem_ext}
  There is no $3$-connected $1$-edge extension of $W_{3}$, so the theorem follows from Corollary 7.2.10.b.
\end{proof}

\begin{theorem}[7.3.3]
  \label{thm:7.3.3}
  \todo[inline]{add}
\end{theorem}

\begin{theorem}[7.3.4]
  \label{thm:7.3.4}
  \todo[inline]{add}
\end{theorem}

\begin{theorem}[7.4.1]
  \label{thm:7.4.1}
  \todo[inline]{add}
\end{theorem}
