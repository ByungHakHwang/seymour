\chapter{Regularity of 2-Sum}

\begin{definition}
    \label{def:two_sum_matrix}
    % \uses{}
    % \lean{}
    \leanok
    Let $R$ be a semiring (we will use $R = \mathbb{Z}_{2}$ and $R = \mathbb{Q}$). Let $B_{\ell} \in R^{(X_{\ell} \cup \{x\}) \times Y_{\ell}}$ and $B_{r} \in R^{X_{r} \times (Y_{r} \cup \{y\})}$ be matrices of the form
    \[
        B_{\ell} = \begin{bmatrix} A_{\ell} \\ r \end{bmatrix}, \quad
        B_{r} = \begin{bmatrix} c & A_{r} \end{bmatrix}.
    \]
    The $2$-sum $B = B_{\ell} \oplus_{2, x, y} B_{r}$ of $B_{\ell}$ and $B_{r}$ is defined as
    \[
        B = \begin{bmatrix} A_{\ell} & 0 \\ D & A_{r} \end{bmatrix}
        \quad \text{where} \quad
        D = c \otimes r.
    \]
    Here $A_{\ell} \in R^{X_{\ell} \times Y_{\ell}}$, $A_{r} \in R^{X_{r} \times Y_{r}}$, $r \in R^{Y_{\ell}}$, $c \in R^{X_{r}}$, $D \in R^{X_{r} \times Y_{\ell}}$, and the indexing is consistent everywhere.
\end{definition}

\begin{definition}
    \label{def:two_sum_matroid}
    \uses{def:std_repr,def:two_sum_matrix}
    % \lean{}
    \leanok
    A matroid $M$ is a $2$-sum of matroids $M_{\ell}$ and $M_{r}$ if there exist standard $\mathbb{Z}_{2}$ representation matrices $B$, $B_{\ell}$, and $B_{r}$ (for $M$, $M_{\ell}$, and $M_{r}$, respectively) of the form given in Definition~\ref{def:two_sum_matrix}.
\end{definition}

\begin{lemma}
    \label{lem:two_sum_bottom_tu}
    \uses{def:two_sum_matrix,def:tu}
    % \lean{}
    \leanok
    Let $B_{\ell}$ and $B_{r}$ from Definition~\ref{def:two_sum_matrix} be TU matrices (over $\mathbb{Q}$). Then $C = \begin{bmatrix} D & A_{r} \end{bmatrix}$ is TU.
\end{lemma}

\begin{proof}
    \uses{def:two_sum_matrix,def:tu}
    \leanok
    Since $B_{\ell}$ is TU, all its entries are in $\{0, \pm 1\}$. In particular, $r$ is a $\{0, \pm 1\}$ vector. Therefore, every column of $D$ is a copy of $y$, $-y$, or the zero column. Thus, $C$ can be obtained from $B_{r}$ by adjoining zero columns, duplicating the $y$ column, and multiplying some columns by $-1$. Since all these operations preserve TUess and since $B_{r}$ is TU, $C$ is also TU.
\end{proof}

\begin{lemma}
    \label{lem:two_sum_pivot}
    \uses{def:two_sum_matrix,def:stp}
    % \lean{}
    \leanok
    Let $B_{\ell}$ and $B_{r}$ be matrices from Definition~\ref{def:two_sum_matrix}. Let $B_{\ell}'$ and $B'$ be the matrices obtained by performing a short tableau pivot on $(x_{\ell}, y_{\ell}) \in X_{\ell} \times Y_{\ell}$ in $B_{\ell}$ and $B$, respectively. Then $B' = B_{\ell}' \oplus_{2, x, y} B_{r}$.
\end{lemma}

\begin{proof}
    \uses{def:two_sum_matrix,def:stp,lem:stp_block_zero}
    \leanok
    Let
    \[
        B_{\ell}' = \begin{bmatrix} A_{\ell}' \\ r' \end{bmatrix}, \quad
        B' = \begin{bmatrix} B_{11}' & B_{12}' \\ B_{21}' & B_{22}' \end{bmatrix}
    \]
    where the blocks have the same dimensions as in $B_{\ell}$ and $B$, respectively. By Lemma~\ref{lem:stp_block_zero}, $B_{11}' = A_{\ell}'$, $B_{12}' = 0$, and $B_{22}' = A_{r}$. Equality $B_{21}' = c \otimes r'$ can be verified via a direct calculation. Thus, $B' = B_{\ell}' \oplus_{2, x, y} B_{r}$.
\end{proof}

\begin{lemma}
    \label{lem:two_sum_tu}
    \uses{def:two_sum_matrix,def:tu}
    % \lean{}
    \leanok
    Let $B_{\ell}$ and $B_{r}$ from Definition~\ref{def:two_sum_matrix} be TU matrices (over $\mathbb{Q}$). Then $B_{\ell} \oplus_{2, x, y} B_{r}$ is TU.
\end{lemma}

\begin{proof}
    \uses{lem:tu_iff_all_pu,def:two_sum_matrix,lem:two_sum_bottom_tu,lem:stp_pn_abs_det_eq,lem:two_sum_pivot}
    \leanok
    By Lemma~\ref{lem:tu_iff_all_pu}, it suffices to show that $B_{\ell} \oplus_{2, x, y} B_{r}$ is $k$-PU for every $k \in \mathbb{Z}_{\geq 1}$. We prove this claim by induction on $k$. The base case with $k = 1$ holds, since all entries of $B_{\ell} \oplus_{2, x, y} B_{r}$ are in $\{0, \pm 1\}$ by construction.

    Suppose that for some $k \in \mathbb{Z}_{\geq 1}$ we know that for any TU matrices $B_{\ell}'$ and $B_{r}'$ (from Definition~\ref{def:two_sum_matrix}) their $2$-sum $B_{\ell}' \oplus_{2, x, y} B_{r}'$ is $k$-PU. Now, given TU matrices $B_{\ell}$ and $B_{r}$ (from Definition~\ref{def:two_sum_matrix}), our goal is to show that $B = B_{\ell} \oplus_{2, x, y} B_{r}$ is $(k + 1)$-PU, i.e., that every $(k + 1) \times (k + 1)$ submatrix $T$ of $B$ has $\det T \in \{0, \pm 1\}$.

    First, suppose that $T$ has no rows in $X_{\ell}$. Then $T$ is a submatrix of $\begin{bmatrix} D & A_{r} \end{bmatrix}$, which is TU by Lemma~\ref{lem:two_sum_bottom_tu}, so $\det T \in \{0, \pm 1\}$. Thus, we may assume that $T$ contains a row $x_{\ell} \in X_{\ell}$.

    Next, note that without loss of generality we may assume that there exists $y_{\ell} \in Y_{\ell}$ such that $T (x_{\ell}, y_{\ell}) \neq 0$. Indeed, if $T (x_{\ell}, y) = 0$ for all $y$, then $\det T = 0$ and we are done, and $T (x_{\ell}, y) = 0$ holds whenever $y \in Y_{r}$.

    Since $B$ is $1$-PU, all entries of $T$ are in $\{0, \pm 1\}$, and hence $T (x_{\ell}, y_{\ell}) \in \{\pm 1\}$. Thus, by Lemma~\ref{lem:stp_pn_abs_det_eq}, performing a short tableau pivot in $T$ on $(x_{\ell}, y_{\ell})$ yields a matrix that contains a $k \times k$ submatrix $T''$ such that $|\det T| = |\det T''|$. Since $T$ is a submatrix of $B$, matrix $T''$ is a submatrix of the matrix $B'$ resulting from performing a short tableau pivot in $B$ on the same entry $(x_{\ell}, y_{\ell})$. By Lemma~\ref{lem:two_sum_pivot}, we have $B' = B_{\ell}' \oplus_{2, x, y} B_{r}$ where $B_{\ell}'$ is the result of performing a short tableau pivot in $B_{\ell}$ on $(x_{\ell}, y_{\ell})$. Since TUness is preserved by pivoting and $B_{\ell}$ is TU, $B_{\ell}'$ is also TU. Thus, by the inductive hypothesis applied to $T''$ and $B_{\ell}' \oplus_{2, x, y} B_{r}$, we have $\det T'' \in \{0, \pm 1\}$. Since $|\det T| = |\det T''|$, we conclude that $\det T \in \{0, \pm 1\}$.
\end{proof}

\begin{theorem}
    \label{thm:two_sum_regular}
    \uses{def:regular,def:two_sum_matroid}
    % \lean{}
    \leanok
    Let $M$ be a $2$-sum of regular matroids $M_{\ell}$ and $M_{r}$. Then $M$ is also regular.
\end{theorem}

\begin{proof}
    \uses{def:std_repr,def:two_sum_matroid,lem:regular_defs_equiv,lem:two_sum_tu,def:tu_signing}
    \leanok
    Let $B$, $B_{\ell}$, and $B_{r}$ be standard $\mathbb{Z}_{2}$ representation matrices from Definition~\ref{def:two_sum_matroid}. Since $M_{\ell}$ and $M_{r}$ are regular, by Lemma~\ref{lem:regular_defs_equiv}, $B_{\ell}$ and $B_{r}$ have TU signings $B_{\ell}'$ and $B_{r}'$, respectively. Then $B' = B_{\ell}' \oplus_{2, x, y} B_{r}'$ is a TU signing of $B$. Indeed, $B'$ is TU by Lemma~\ref{lem:two_sum_tu}, and a direct calculation verifies that $B'$ is a signing of $B$. Thus, $M$ is regular by Lemma~\ref{lem:regular_defs_equiv}.
\end{proof}
