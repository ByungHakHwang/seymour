\chapter{Special Matroids}

\begin{definition}
    \label{Matrix.IsGraphic}
    \uses{Matrix}
    \leanok
    Let $A \in \mathbb{Q}^{X \times Y}$ be a matrix. If for all $j \in Y$, one has that $a_{i,j} = 0$ for all $i \in X$, or that there exists $i_1,i_2 \in X$ such that
    \[
    a_{i,j} = \begin{cases}
        1 & \text{ if $i = i_1$} \\
        -1 & \text{ if $i = i_2$} \\
        0 & \text{ otherwise},
    \end{cases}
    \]
    then we call $A$ a node-incidence matrix for a (directed) graph whose nodes are indexed by $X$ and whose edges are indexed by $Y$.
\end{definition}

\begin{definition}
    \label{Matroid.IsGraphic}
    \uses{VectorMatroid,Matrix.IsGraphic}
    \leanok
    We say that a matroid is graphic if it can be represented by a node-incidence matrix.
\end{definition}

\begin{definition}
    \label{StandardRepr.dual}
    \uses{StandardRepr}
    \leanok
    Let $S$ be a standard representation given by matrix $B$. The dual of $S$ is given by $-B^\intercal$.
\end{definition}

\begin{definition}
    \label{Matroid.IsCographic}
    \uses{Matroid.IsGraphic,StandardRepr.dual}
    \leanok
    We say a matroid is co-graphic if its dual is graphic.
\end{definition}

\begin{definition}
    \label{matroidR10}
    \uses{StandardRepr}
    \leanok
    The matroid with standard representation
        \[\begin{bmatrix}
            1 & 0 & 0 & 1 & 1 \\
            1 & 1 & 0 & 0 & 1 \\
            0 & 1 & 1 & 0 & 1 \\
            0 & 0 & 1 & 1 & 1 \\
            1 & 1 & 1 & 1 & 1 \\
        \end{bmatrix}\]
        over $\mathbb{Z}_2$ is called $R_{10}.$
    \end{definition}

\begin{theorem}
    \label{matroidR10.isRegular}
    \uses{matroidR10,Matroid.IsRegular}
    \leanok
    The matroid $R_{10}$ is regular.
\end{theorem}

\begin{proof}
    \leanok
    See Lean implementation.
\end{proof}

\begin{theorem}
    \label{Matroid.IsGraphic.isRegular}
    \uses{Matroid.IsGraphic,Matroid.IsRegular}
    \leanok
    Every graphic matroid is regular.
\end{theorem}

\begin{proof}
    \leanok
    See Lean implementation.
\end{proof}
