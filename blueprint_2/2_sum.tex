\section{Regularity of 2-Sum}

\subsection{Preliminaries}

\begin{lemma}\label{lem:pivot_submatrix_det}
    Let $A$ be a $k \times k$ matrix. Let $r, c \in \{1, \dots, k\}$ be a row and column index, respectively, such that $a_{rc} \neq 0$. Let $A'$ denote the matrix obtained from $A$ by performing a real pivot on $a_{rc}$. Then there exists a $(k - 1) \times (k - 1)$ submatrix $A''$ of $A'$ with $|\det A''| = \frac{|\det A|}{|a_{rc}|}$.
\end{lemma}

\begin{proof}
    Let $A''$ be the submatrix of $A'$ given by row index set $R = \{1, \dots, k\} \setminus \{r\}$ and column index set $C = \{1, \dots, k\} \setminus \{c\}$. By the explicit formula for pivoting in $A$ on $a_{rc}$, the entries of $A''$ are given by $a_{ij}'' = a_{ij} - \frac{a_{ic} \cdot a_{rj}}{a_{rc}}$. Using the linearity of the determinant, we can express $\det A''$ as
    \[
        \det A'' = \det A' - \sum_{k \in C} \frac{a_{rk}}{a_{rc}} \cdot \det B_{k}''
    \]
    where $B_{k}''$ is a matrix obtained from $A''$ by replacing column $a_{\bullet k}''$ with the pivot column $a_{\bullet c}$ without the pivot element $a_{rc}$.

    By the cofactor expansion in $A$ along row $r$, we have
    \[
        \det A = \sum_{k = 1}^{n} (-1)^{r + k} \cdot a_{rk} \cdot \det B_{r, k}
    \]
    where $B_{r, k}$ is obtained from $A$ by removing row $r$ and column $k$. By swapping the order of columns in $B_{r, k}$ to match the form of $B_{k}$, we get
    \[
        \det A = (-1)^{r + c} (a_{rc} \cdot \det A' - \sum_{k \in C} a_{rk} \cdot \det B_{k}'').
    \]

    By combining the above results, we get $|\det A''| = \frac{|\det A|}{|a_{rc}|}$.
\end{proof}

\begin{corollary}\label{cor:pivot_submatrix_det}
    Let $A$ be a $k \times k$ matrix with $\det A \notin \{0, \pm 1\}$. Let $r, c \in \{1, \dots, k\}$ be a row and column index, respectively, and suppose that $a_{rc} \in \{\pm 1\}$. Let $A'$ denote the matrix obtained from $A$ by performing a real pivot on $a_{rc}$. Then there exists a $(k - 1) \times (k - 1)$ submatrix $A''$ of $A'$ with $\det A'' \notin \{0, \pm 1\}$.
\end{corollary}

\begin{proof}
    Since $a_{rc} \in \{\pm 1\}$, by Lemma~\ref{lem:pivot_submatrix_det} there exists a $(k - 1) \times (k - 1)$ submatrix $A''$ with $|\det A| = |\det A''|$. Since $\det A \notin \{0, \pm 1\}$, we have $\det A'' \notin \{0, \pm 1\}$.
\end{proof}

\begin{definition}\label{def:k_tu}
    Given $k \in \mathbb{Z}_{\geq 1}$, we say that a matrix $A$ is $k$-TU if every square submatrix of $A$ of size $k$ has determinant in $\{0, \pm 1\}$.
\end{definition}

\begin{remark}
    Note that a matrix is TU if and only if it is $k$-TU for every $k \in \mathbb{Z}_{\geq 1}$.
\end{remark}


\subsection{Proof of Regularity}

\begin{definition}\label{def:two_sum}
    Let $B_{l}, B_{r}$ be matrices with $\{0, \pm 1\}$ entries expressed as $B_{l} = \begin{NiceArray}{c}[hvlines] A_{l} \\ x \\ \end{NiceArray}$ and $B_{r} = \begin{NiceArray}{cc}[hvlines] y & A_{r} \\ \end{NiceArray}$, where $x$ is a row vector, $y$ is a column vector, and $A_{l}, A_{r}$ are matrices of appropriate dimensions. Let $D$ be the outer product of $y$ and $x$. The $2$-sum of $B_{l}$ and $B_{r}$ is defined as
    \[
        B_{l} \oplus_{2, x, y} B_{r} = \begin{NiceArray}{cc}[hvlines] A_{l} & 0 \\ D & A_{r} \\ \end{NiceArray}.
    \]
\end{definition}

\begin{lemma}\label{lem:two_sum_1_2_tu}
    Let $B_{l}$ and $B_{r}$ be TU matrices and let $B = B_{l} \oplus_{2, x, y} B_{r}$. Then $B$ is $1$-TU and $2$-TU.
\end{lemma}

\begin{proof}
    To see that $B$ is $1$-TU, note that $B$ is a $\{0, \pm 1\}$ matrix by construction.

    To show that $B$ is $2$-TU, let $V$ be a $2 \times 2$ submatrix $V$ of $B$. If $V$ is a submatrix of $\begin{NiceArray}{c}[hvlines] A_{l} \\ D \\ \end{NiceArray}$, $\begin{NiceArray}{cc}[hvlines] D & A_r \end{NiceArray}$, $\begin{NiceArray}{cc}[hvlines] A_{l} & 0 \end{NiceArray}$, or  $\begin{NiceArray}{c}[hvlines] 0 \\ A_r \\ \end{NiceArray}$, then $\det V \in \{0, \pm 1\}$, as all of those four matrices are TU. Otherwise $V$ shares exactly one row and one column index with both $A_{l}$ and $A_{r}$. Let $i$ be the row shared by $V$ and $A_{l}$ and $j$ be the column shared by $V$ and $A_{r}$. Note that $V_{ij} = 0$. Thus, up to sign, $\det V$ equals the product of the entries on the diagonal not containing $V_{ij}$. Since both of those entries are in $\{0, \pm 1\}$, we have $\det V \in \{0, \pm 1\}$.
\end{proof}

\begin{lemma}\label{lem:two_sum_k_tu_induction}
    Let $k \in \mathbb{Z}_{\geq 1}$. Suppose that for any TU matrices $B_{l}$ and $B_{r}$ their $2$-sum $B = B_{l} \oplus_{2, x, y} B_{r}$ is $\ell$-TU for every $\ell < k$. Then for any TU matrices $B_{l}$ and $B_{r}$ their $2$-sum $B = B_{l} \oplus_{2, x, y} B_{r}$ is also $k$-TU.
\end{lemma}

\begin{proof}
    For the sake of deriving a contradiction, suppose there exist TU matrices $B_{l}$ and $B_{r}$ such that their $2$-sum $B = B_{l} \oplus_{2, x, y} B_{r}$ is not $k$-TU. Then $B$ contains a $k \times k$ submatrix $V$ with $\det V \notin \{0, \pm 1\}$.

    Note that $V$ cannot be a submatrix of $\begin{NiceArray}{c}[hvlines] A_{l} \\ D \\ \end{NiceArray}$, $\begin{NiceArray}{cc}[hvlines] D & A_r \end{NiceArray}$, $\begin{NiceArray}{cc}[hvlines] A_{l} & 0 \end{NiceArray}$, or  $\begin{NiceArray}{c}[hvlines] 0 \\ A_r \\ \end{NiceArray}$, as all of those four matrices are TU. Thus, $V$ shares at least one row and one column index with $A_{l}$ and $A_{r}$ each.

    Consider the row of $V$ whose index appears in $A_{l}$. Note that it cannot consist of only $0$ entries, as otherwise $\det V = 0$. Thus there exists a $\pm 1$ entry shared by $V$ and $A_{l}$. Let $r$ and $c$ denote the row and column index of this entry, respectively.

    Perform a rational pivot in $B$ on the element $B_{rc}$. For every object, its modified counterpart after pivoting is denoted by the same symbol with an added tilde; for example, $\tilde{B}$ denotes the entire matrix after the pivot. Note that after pivoting the following statements hold:
    \begin{itemize}
        \item $\begin{NiceArray}{c}[hvlines] \rule{0pt}{1.15em}\tilde{A}_{l} \\ \rule{0pt}{1.15em}\tilde{D} \\ \end{NiceArray}$ is TU, since TUness is preserved by pivoting.
        \item $\tilde{A}_{2} = A_{r}$, i.e., $A_{r}$ remains unchanged. This holds because of the $0$ block in $B$.
        \item $\tilde{D}$ consists of copies of $y$ scaled by factors in $\{0, \pm 1\}$. This can be verified via a case distinction and a simple calculation.
        \item $\begin{NiceArray}{cc}[hvlines] \rule{0pt}{1.15em}\tilde{D} & \rule{0pt}{1.15em}\tilde{A_r} \end{NiceArray}$ is TU, since this matrix consists of $A_{r}$ and copies of $y$ scaled by factors $\{0, \pm 1\}$.
        \item $\tilde{D}$ can be represented as an outer product of a column vector $\tilde{y}$ and a row vector $\tilde{x}$, and we can define $\tilde{B}_{1} = \begin{NiceArray}{c}[hvlines] \rule{0pt}{1em}\tilde{A_l} \\ \rule{0pt}{1.15em}\tilde{x} \\ \end{NiceArray}$ and $\tilde{B}_{2} =\begin{NiceArray}{cc}[hvlines] \rule{0pt}{1em}\tilde{y} & \rule{0pt}{1.15em}\tilde{A_r} \end{NiceArray}$ similar to $B_{l}$ and $B_{r}$, respectively. Note that $\tilde{B}_{1}$ and $\tilde{B}_{2}$ have the same size as $B_{l}$ and $B_{r}$, respectively, are both TU, and satisfy $\tilde{B} = \tilde{B}_{1} \oplus_{2, \tilde{x}, \tilde{y}} \tilde{B}_{2}$.
        \item $\tilde{B}$ contains a square submatrix $\tilde{V}$ of size $k - 1$ with $\det \tilde{V} \notin \{0, \pm 1\}$. Indeed, by Corollary~\ref{cor:pivot_submatrix_det} from Lemma~\ref{lem:pivot_submatrix_det}, pivoting in $V$ on the element $B_{rc}$ results in a matrix containing a $(k - 1) \times (k - 1)$ submatrix $V''$ with $\det V'' \in \{0, \pm 1\}$. Since $V$ is a submatrix of $B$, the submatrix $V''$ corresponds to a submatrix $\tilde{V}$ of $\tilde{B}$ with the same property.
    \end{itemize}
    To sum up, after pivoting we obtain a matrix $\tilde{B}$ that represents a $2$-sum of TU matrices $\tilde{B}_{1}$ and $\tilde{B}_{2}$ and contains a square submatrix of size $k - 1$ with determinant not in $\{0, \pm 1\}$. This is a contradiction with $(k - 1)$-TUness of $\tilde{B}$, which proves the lemma.
\end{proof}

\begin{lemma}\label{lem:two_sum_tu}
    Let $B_{l}$ and $B_{r}$ be TU matrices. Then $B_{l} \oplus_{2, x, y} B_{r}$ is also TU.
\end{lemma}

\begin{proof}
    Proof by induction.

    Proposition for any $k \in \mathbb{Z}_{\geq 1}$: For any TU matrices $B_{l}$ and $B_{r}$, their $2$-sum $B = B_{l} \oplus_{2, x, y} B_{r}$ is $\ell$-TU for every $\ell \leq k$.

    Base: The Proposition holds for $k = 1$ and $k = 2$ by Lemma~\ref{lem:two_sum_1_2_tu}.

    Step: If the Proposition holds for some $k$, then it also holds for $k + 1$ by Lemma~\ref{lem:two_sum_k_tu_induction}.

    Conclusion: For any TU matrices $B_{l}$ and $B_{r}$, their $2$-sum $B_{l} \oplus_{2, x, y} B_{r}$ is $k$-TU for every $k \in \mathbb{Z}_{\geq 1}$. Thus, $B_{l} \oplus_{2, x, y} B_{r}$ is TU.
\end{proof}
