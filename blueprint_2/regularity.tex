\section{Equivalence of Definitions of Regularity}

\subsection{Support Matrices and Their Properties}

\begin{definition}\label{def:support_matrix}
    Let $F$ be a field. The support of matrix $A \in F^{X \times Y}$ is $A^{\#} \in \{0, 1\}^{X \times Y}$ given by
    \[
        \forall i \in X, \ \forall j \in Y, \ A^{\#}_{i, j} = \begin{cases}
            0, & \text{ if } A_{i, j} = 0, \\
            1, & \text{ if } A_{i, j} \neq 0.
        \end{cases}
    \]
\end{definition}

\begin{definition}\label{def:fundamental_circuit}
    Let $M$ be a matroid, let $B$ be a base of $M$, and let $e \in E \setminus B$ be an element. The fundamental circuit $C (e, B)$ of $e$ with respect to $B$ is the unique circuit contained in $B \cup \{e\}$.
\end{definition}

\begin{lemma}\label{lem:std_repr_fundamental_circuits}
    Let $M$ be a matroid and let $S \in F^{X \times Y}$ be a standard representation matrix of $M$ over a field $F$. Then $\forall y \in Y$, the fundamental circuit of $y$ w.r.t. $X$ is $C (y, X) = \{y\} \cup \{x \in X \mid S (x, y) \neq 0\}$.
\end{lemma}

\begin{proof}
    Let $y \in Y$. Our goal is to show that $C' (y, X) = \{y\} \cup \{x \in X \mid D (x, y) \neq 0\}$ is a fundamental circuit of $y$ with respect to $X$.
    \begin{itemize}
        \item $C' (y, X) \subseteq X \cup \{y\}$ by construction.
        \item $C' (y, X)$ is dependent, since columns of $[I \mid S]$ indexed by elements of $C (y, X)$ are linearly dependent.
        \item If $C \subsetneq C' (y, X)$, then $C$ is independent. To show this, let $V$ be the set of columns of $[I \mid S]$ indexed by elements of $C$ and consider two cases.
        \begin{enumerate}
            \item Suppose that $y \notin C$. Then vectors in $V$ are linearly independent (as columns of $I$). Thus, $C$ is independent.
            \item Suppose $\exists x \in X \setminus C$ such that $S (x, y) \neq 0$. Then any nontrivial linear combination of vectors in $V$ has a non-zero entry in row $x$. Thus, these vectors are linearly independent, so $C$ is independent.
        \end{enumerate}
    \end{itemize}
\end{proof}

\begin{lemma}\label{lem:std_repr_support_matrix_cols}
    Let $M$ be a matroid and let $S \in F^{X \times Y}$ be a standard representation matrix of $M$ over a field $F$. Then $\forall y \in Y$, column $S^{\#} (\bullet, y)$ is the characteristic vector of $C (y, X) \setminus \{y\}$.
\end{lemma}

\begin{proof}
    This directly follows from Lemma~\ref{lem:std_repr_fundamental_circuits}.
\end{proof}

\begin{lemma}\label{lem:repr_TU_iff_repr_TU_support}
    Let $A$ be a TU matrix.
    \begin{enumerate}
        \item If a matroid is represented by $A$, then it is also represented by $A^{\#}$.
        \item If a matroid is represented by $A^{\#}$, then it is also represented by $A$.
    \end{enumerate}
\end{lemma}

\begin{proof}
    See Lean implementation.\todo{need details?}
\end{proof}


\subsection{Conversion from General to Standard Representation}

\begin{lemma}\label{lem:repr_to_std_repr}
    Let $M$ be a matroid represented by a matrix $A \in \mathbb{Q}^{X \times Y}$ and let $B$ be a base of $M$. Then there exists a matrix $S \in \mathbb{Q}^{B \times (Y \setminus B)}$ that is a standard representation matrix of $M$.
\end{lemma}

\begin{proof}
    Let $C = \{A (\bullet, b) \mid b \in B\}$. Since $B$ is a base of $M$, we can show that $C$ is a basis in the column space $\mathrm{span} \{A (\bullet, y) \mid y \in Y\}$. For every $y \in Y \setminus B$, let $S (\bullet, y)$ be the coordinates of $A (\bullet, y)$ in basis $C$. We can show that $[I \mid S]$ represents the same matroid as $A$, so $S$ is a standard representation matrix of $M$.\todo{see details in implementation}
\end{proof}

\begin{lemma}\label{lem:TU_repr_to_TU_std_repr}
    Let $M$ be a matroid represented by a TU matrix $A \in \mathbb{Q}^{X \times Y}$ and let $B$ be a base of $M$. Then there exists a matrix $S \in \mathbb{Q}^{B \times (Y \setminus B)}$ such that $S$ is TU and $S$ is a standard representation matrix of $M$.
\end{lemma}

\begin{proof}[Proof sketch.]
    Apply the procedure described in the proof of Lemma~\ref{lem:repr_to_std_repr} to $A$. This procedure can be represented as a sequence of elementary row operations, all of which preserve TUness. Dropping the identity matrix at the end also preserves TUness.
    \todo[inline]{formalize}
\end{proof}


\subsection{Proof of Equivalence}

\begin{definition}\label{def:regular}
    A matroid $M$ is regular if $\exists A \in \mathbb{Q}^{X \times Y}$ such that $A$ is TU and $A$ represents $M$.
\end{definition}

\begin{definition}\label{def:tu_signing}
    We say that $A' \in \mathbb{Q}^{X \times Y}$ is a TU signing of $A \in \mathbb{Z}_{2}^{X \times Y}$ if $A'$ is TU and
    \[
        \forall i \in X, \ \forall j \in Y, \ |A_{i, j}'| = A_{i, j}.
    \]
\end{definition}

\begin{lemma}\label{lem:regular_defs_equiv}
    Let $M$ be a matroid given by a standard representation matrix $B \in \mathbb{Z}_{2}^{X \times Y}$. Then the following are equivalent.
    \begin{enumerate}
        \item\label{item:regular_defs_equiv_regular} $M$ is regular.
        \item\label{item:regular_defs_equiv_tu_signing} $B$ has a TU signing.
    \end{enumerate}
\end{lemma}

\begin{proof}
    ~
    \begin{itemize}
        \item[\ref{item:regular_defs_equiv_regular} $\Rightarrow$ \ref{item:regular_defs_equiv_tu_signing}] Recall that $X$ (the row set of $B$) is a base of $M$. By Lemma~\ref{lem:TU_repr_to_TU_std_repr}, $A$ can be converted into a standard representation matrix $B' \in \mathbb{Q}^{X \times Y}$ of $M$ such that $B'$ is also TU. Since $B'$ and $B$ are both standard representations of $M$, by Lemma~\ref{lem:std_repr_support_matrix_cols} the support matrices $(B')^{\#}$ and $B^{\#}$ are the same. Moreover, $B^{\#} = B$, since $B$ has entries in $\mathbb{Z}_{2}$. Thus, $B'$ is TU and $(B')^{\#} = B$, so $B'$ is a TU signing of $B$.

        \item[\ref{item:regular_defs_equiv_tu_signing} $\Rightarrow$ \ref{item:regular_defs_equiv_regular}] Let $B' \in \mathbb{Q}^{X \times Y}$ be a TU signing of $B$. Then $A = [I \mid B']$ is TU, as it is obtained from $B'$ by adjoining the identity matrix. Moreover, by Lemma~\ref{lem:repr_TU_iff_repr_TU_support}, $A$ represents the same matroid as $A^{\#} = [I \mid B]$, which is $M$. Thus, $A$ is a TU matrix representing $M$, so $M$ is regular.
    \end{itemize}
\end{proof}
