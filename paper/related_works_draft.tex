\documentclass[sigplan,10pt,anonymous,review]{acmart}

\settopmatter{printfolios=true,printccs=false,printacmref=false}

\usepackage{graphicx} % Required for inserting images
\usepackage[T1]{fontenc}
\usepackage[utf8]{inputenc}
\usepackage{listings}
\usepackage{amsmath}
\usepackage{amssymb}

\usepackage{color}
\definecolor{keywordcolor}{rgb}{0.7, 0.1, 0.1}   % red
\definecolor{tacticcolor}{rgb}{0.0, 0.1, 0.6}    % blue
\definecolor{commentcolor}{rgb}{0.4, 0.4, 0.4}   % grey
\definecolor{symbolcolor}{rgb}{0.0, 0.1, 0.6}    % blue
\definecolor{sortcolor}{rgb}{0.1, 0.5, 0.1}      % green
\definecolor{attributecolor}{rgb}{0.7, 0.1, 0.1} % red

\def\lstlanguagefiles{lstlean.tex}
\lstset{language=lean} % set default language

\title{Related Works - Draft}
\newtheorem{theorem}{Theorem}[section]

\begin{document}

	\maketitle

	% This is a VERY rough draft. I expect that much revision will be needed!

	\section{Related Works}
	There exist repositories due to \href{https://github.com/VArtem/lean-matroids}{Artem Vasilyev} and \href{https://github.com/bryangingechen/lean-matroids}{Bryan Gin-ge Chen}
	dedicated to formalizing some of the theory of finite-rank matroids in Lean 3. As is standard in the literature, both of these formalizations follow the definition of a
	finite-rank matroid as given in the first chapter of \textit{Matroid Theory} by John Oxley. % clarify standard definition of finite rank matroid?
	In particular, this means that they define bases of matroids separately and then prove that they exist, are maximal, and satisfy the exchange property.
	In contrast, the definition of matroids used in this paper is the current definition in \texttt{mathlib4}, which takes the definition of an infinite matrix from the
	Bruhn et al.'s paper \textit{Axioms of Infinite Matroids}. Here, the notion of a base is baked in along with the requirement that the set of all bases satisfies the
	maximality and exchange properties. From this point, Chen and Vasilyev both prove basic properties about matroids, circuits, bases, and rank functions.
	Namely they both show that the rank function of a matroid satisfies the monotonicity and submodularity properties.

	Alena Gusakov's \href{https://dspacemainprd01.lib.uwaterloo.ca/server/api/core/bitstreams/fe5957ef-3e10-4493-b11d-8d8121cafeba/content}{master's thesis}
	implements the \texttt{mathlib4} definition of a matroid due to Bruhn et al.\ as a basis for formalizing and proving facts about matroids,
	matroid minors, and their representability. In particular, Gusakov formalizes the statement and proof of Tutte's 1985 excluded minor characterization of binary matroids:
	\begin{quote}
		A matroid is binary if and only if it has no $U_{2,4}$-minor.
	\end{quote}

	Using the interactive theorem prover Isabelle/HOL, \href{https://link.springer.com/chapter/10.1007/978-981-96-1621-3_18}{Wan et al.}
	formalize the standard notion of a matroid of finite rank (using a collection of independent sets) and create a ``mechanized verification framework''
	for judging whether or not a pair $(S, L)$ with $S$ a set and $L$ a set of subsets of $S$ is a matroid by using a Locale. The authors then use
	this as a way to model the 0-1 knapsack problem, which can be stated as follows:
	\begin{quote}
		Suppose one has $n$ indivisible items each of of weight $w_i$ and value $v_i$ where $i \in \{1,\ldots, n\}$ and a knapsack of capacity $w$.
		How does one maximize $\sum_{i=1}^n v_i$ subject to the constraint $\sum_{i=1}^n w_i \le w$?
	\end{quote}
	They also formalise the so-called ``fractional knapsack problem'' in which the indivisibility requirement on the items in question is removed.

	\section{Standard Representation}
	The \emph{standard representation} of a vector matroid is a structure that couples a \emph{standard representation matrix} with some other data. It is defined over two types, $\alpha$ and $R$. The additional data consists of a row index set and a column index sets, both of which contain elements of type $\alpha$, it includes a proof that these two sets are disjoint, and two instances that say a computer can determine whether certain element is a row or a column, i.e. membership of an $\alpha$-typed term onto the row (or column) index set is decidable. These instances can be ignored when working in a classical framework, which the Seymour project occurs in. The standard representation matrix is an $X \times Y$-matrix.

	\begin{lstlisting}
		/-- Standard matrix representation of a vector matroid. -/
		structure StandardRepr (α R : Type) [DecidableEq α] where
		/-- Row indices. -/
		X : Set α
		/-- Col indices. -/
		Y : Set α
		/-- Basis and nonbasis elements are disjoint -/
		hXY : X ⫗ Y
		/-- Standard representation matrix. -/
		B : Matrix X Y R
		/-- The computer can determine whether certain element is a row. -/
		decmemX : ∀ a, Decidable (a ∈ X)
		/-- The computer can determine whether certain element is a col. -/
		decmemY : ∀ a, Decidable (a ∈ Y)
	\end{lstlisting}


	The point of introducing such a structure is that elements of $X$ will act as a basis while elements of $Y$ act as the non-basis. This will make manipulations easier while also allowing us to change it to a "full representation" easily, which is a $X \times \left(X \cup Y\right)$-matrix.

	\begin{lstlisting}
		/-- Convert standard representation of a vector matroid to a full representation. -/
		def StandardRepr.toFull [Zero R] [One R] (S : StandardRepr α R) : Matrix S.X (S.X ∪ S.Y).Elem R :=
		((1 ◫ S.B) · ∘ Subtype.toSum)
	\end{lstlisting}

	We can convert a standard representation into a matroid by converting it into a full representation first, then converting that matrix into a matroid.

	\begin{lstlisting}
		/-- Construct a matroid from a standard representation. -/
		def StandardRepr.toMatroid [DivisionRing R] (S : StandardRepr α R) : Matroid α :=
		S.toFull.toMatroid
	\end{lstlisting}

	This matroid will have $X \cup Y$ as a ground set, and $X$ is a base in it. Given a set $I$ of $\alpha$-typed elements, $I$ is independent over this matroid if and only if $I \subset X \cup Y$ and the set of vectors resultant from taking the columns indexed by $I$ of the matrix formed from gluing the identity matrix $I_\text{card}\left(X\right)$ to the left of the standard representation matrix is linearly independent over $R$.

	The most complicated proof of this file is of the following statement:
	A matroid from a totally unimodular full representation will admit a totally unimodular standard representation that gives the same matroid.

	\begin{lstlisting}
		lemma Matrix.exists_standardRepr_isBase_isTotallyUnimodular [Field R] {G : Set α} [Fintype G] {X Y : Set α}
		(A : Matrix X Y R) (hAG : A.toMatroid.IsBase G) (hA : A.IsTotallyUnimodular) :
		∃ S : StandardRepr α R, S.X = G ∧ S.toMatroid = A.toMatroid ∧ S.B.IsTotallyUnimodular
	\end{lstlisting}

	Moreover, we show that if two standard representations of the same matroid have the same base, then the standard representation matrices have the same support.

	\begin{lstlisting}
		lemma support_eq_support_of_same_matroid_same_X {F₁ F₂ : Type} [Field F₁] [Field F₂] [DecidableEq F₁] [DecidableEq F₂]
		{S₁ : StandardRepr α F₁} {S₂ : StandardRepr α F₂} [Fintype S₂.X]
		(hSS : S₁.toMatroid = S₂.toMatroid) (hXX : S₁.X = S₂.X) :
		let hYY : S₁.Y = S₂.Y := right_eq_right_of_union_eq_union hXX S₁.hXY S₂.hXY (congr_arg Matroid.E hSS)
		hXX ▸ hYY ▸ S₁.B.support = S₂.B.support
	\end{lstlisting}

\end{document}
