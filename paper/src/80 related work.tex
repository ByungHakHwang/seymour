\section{Related Work}

In Lean 4, the largest library formalizing matroid theory is Peter Nelson's repository\footnote{\url{https://github.com/apnelson1/lean-matroids}}. It implements infinite matroids following \cite{Bruhn2013} together with many key notions and results about them. The definition that is fully formalized and is the most related to our work is \texttt{Matroid.disjointSum}. For binary matroids, this definition is equivalent to the 1-sum implemented in this paper. Moreover, it can be used for any matroids with disjoint ground sets, while our implementation is restricted to vector matroids constructed from $\mathbb{Z}_{2}$ matrices. Peter Nelson's repository also makes progress towards formalizing other related notions, such as representable matroids, though this work is still ongoing. It is also worth noting that the results in Mathlib\footnote{\url{https://github.com/leanprover-community/mathlib4}} have been copied over from this repository and comprise a strict subset of it.

Building upon Peter Nelson's work, \citeauthor{Gusakov2024}'s thesis \cite{Gusakov2024} formalizes the proof of Tutte's excluded minor theorem and to this end implements definitions and results about representable matroids. The thesis formalizes representations and standard representations of matroids, which we also do in our work, but it takes a different approach. In particular, instead of working with matrix representations, the thesis implements a representation of \texttt{Matroid $\alpha$} as a mapping from the entire type $\alpha$ to a vector space which maps non-elements of the matroid to the zero vector and independent sets to linearly independent vectors. The advantage of this approach is that certain proofs become easier to formalize, but this comes at a cost of making it harder to match the implementation with the theory and believe the correctness of the code.

There are also two Lean 3 repositories due to \href{https://github.com/VArtem/lean-matroids}{Artem Vasilyev} and \href{https://github.com/bryangingechen/lean-matroids}{Bryan Gin-ge Chen} dedicated to formalization of matroid theory. Both of them work with finite matroids following \cite{Oxley2011} and implement basic definitions and properties of matroids concerning circuits, bases, and rank functions. These results are completely subsumed by the current implementation of matroids in Mathlib.

Jonas Keinholz~\cite{Matroids-AFP} formalizes the classical definition of (finite) matroids \cite{Oxley2011, Truemper2016} in Isabelle/HOL along with other basic ideas such as minors, bases, circuits, rank, and closure. 
More recently, \citeauthor{Wan2025} use Keinholz's formalization to design a verification framework using a Locale that checks if a given collection of subsets of a given set is a matroid. The authors then showcase the verification algorithm by checking that the 0-1 knapsack problem does not conform to the matroid structure, while the fractional knapsack problem does. In comparison, Lean 4's Mathlib implements a more general definition of matroids and formalizes more results about them than either \citeauthor{Matroids-AFP} or \citeauthor{Wan2025}, but Lean lacks a procedure for formally verifying if a collection of sets has matroid structure. % Nevertheless, our repository includes a formally verified algorithm for checking if a matrix is totally unimodular, which is implemented as \texttt{Matrix.testTotallyUnimodular} and can be invoked using the \texttt{\#eval} command.

In the HOL Light GitHub repository\footnote{\url{https://github.com/jrh13/hol-light/blob/master/Library/matroids.ml}}, John Harrison formalizes finitary matroids. The formalization closely follows the field theory notes of Pete L. Clark\footnote{\url{https://plclark.github.io/PeteLClark/Expositions/FieldTheory.pdf}}. In particular, finitary matroids are defined in terms of a closure operator with similar properties as those proposed in \cite{Bruhn2013}. This repository also includes a formal proof that this notion of (finitary) matroids is equivalent to the definition of a matroid using independent sets. Unlike Lean 4's Mathlib formalization (which includes formalizations of the closure operator and the notions of spanning sets), however, this notion of infinite matroids does not respect the notion of duality that is defined for matroids in \cite{Oxley2011,Truemper2016} as noted by \citeauthor{Bruhn2013}.

Grzegorz Bancerek and Yasunari Shidama~\cite{matroid0} formalize matroids in Mizar. Their formalization includes basic notions like rank, basis, and cycle as well as examples like the matroid of linearly independent subsets for a given vector space. Overall, the scope of the Mizar formalization is comparable to the Isabelle/HOL formalization, except that the Mizar formalization allows for infinite matroids. In this sense, it is comparable to the Lean definition in Mathlib, which also allows for infinite matroids. However, whereas Mizar uses independence axioms to define matroids, Lean uses base axioms for the main definition and provides an API for constructing matroids via independence axioms.
