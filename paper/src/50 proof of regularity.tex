\section{Sums Preserve Regularity}

% Final theorems
In our library, the final theorems that regularity is preserved under 1-, 2-, and 3-sums are stated as follows.
% Discussion: \url{https://leanprover.zulipchat.com/\#narrow/channel/113488-general/topic/How.20to.20best.20recall.20P.20.E2.88.A7.20Q.20.E2.86.92.20R/with/532666812
\begin{leancode}
theorem Matroid.IsSum1of.isRegular {α : Type*}
    [DecidableEq α] {M Mₗ Mᵣ : Matroid α} :
  M.IsSum1of Mₗ Mᵣ → M.RankFinite →
    Mₗ.IsRegular → Mᵣ.IsRegular → M.IsRegular
\end{leancode}
\begin{leancode}
theorem Matroid.IsSum2of.isRegular {α : Type*}
    [DecidableEq α] {M Mₗ Mᵣ : Matroid α} :
  M.IsSum2of Mₗ Mᵣ → M.RankFinite →
    Mₗ.IsRegular → Mᵣ.IsRegular → M.IsRegular
\end{leancode}
\begin{leancode}
theorem Matroid.IsSum3of.isRegular {α : Type*}
    [DecidableEq α] {M Mₗ Mᵣ : Matroid α} :
  M.IsSum3of Mₗ Mᵣ → M.RankFinite →
    Mₗ.IsRegular → Mᵣ.IsRegular → M.IsRegular
\end{leancode}
Note that these three theorems are stated for matroids and have the same interface. Moreover, when applying one of these results, a user is able to provide different representations for witnessing that $M$ is a 1-, 2-, or 3-sum of $M_{\ell}$ and $M_{r}$, for witnessing that $M$ has finite rank, and for witnessing that $M_{\ell}$ and $M_{r}$ are regular.% as opposed to the lemmas about the StandardRepr level % Martin: Is it clear with out stating explicitly?

% High-level breakdown
We split the proof of each of these theorems into three stages corresponding to the three abstraction layers used for the definitions: \texttt{Matroid}, \texttt{StandardRepr}, and \texttt{Matrix}. 

The final matroid-level theorems are reduced to the respective lemmas for standard representations by applying \texttt{StandardRepr.toMatroid\_isRegular\_iff\_hasTuSigning} and \texttt{StandardRepr.finite\_X\_of\_toMatroid\_rankFinite} in all three proofs (for the 1-, 2-, and 3-sums). The reduction from the standard representation level to the matrix level for 1- and 2-sums is straightforward\EmDash plug the standard representation matrices and their (rational) signings into \texttt{matrixSum1} and \texttt{matrixSum2}, respectively. For 3-sums, this reduction is more involved, as we additionally apply the following lemma to simplify the assumption on $D_{0}$:
\begin{leancode}
lemma Matrix.isUnit_2x2
    (A : Matrix (Fin 2) (Fin 2) Z2)
    (hA : IsUnit A) :
  ∃ f : Fin 2 ≃ Fin 2,
  ∃ g : Fin 2 ≃ Fin 2,
    A.submatrix f g = 1 ∨
    A.submatrix f g = !![1, 1; 0, 1]
\end{leancode}
Therefore, up to reindexing, $D_{0}$ is either $\begin{bmatrix} 1 & 0 \\ 0 & 1 \\ \end{bmatrix}$ or $\begin{bmatrix} 1 & 0 \\ 1 & 1 \\ \end{bmatrix}$. Performing the reduction at this stage allows us to invoke \texttt{Matrix.isUnit\_2x2} only once and then simply consider the two special forms of $D_{0}$.

% Matrix-level Implementation for 1-Sums
On the matrix level, our formal proof that 1-sums preserve total unimodularity of matrices is nearly identical to \cite{Truemper2016}. For 2-sums, we streamlined the proof by reformulating it as a forward argument by induction. For 3-sums, the entire argument was significantly reworked to simplify and streamline the approach of \cite{Truemper2016}. On a high level, we make two major changes, which we discuss in detail below.

% Only one re-signing: details
The first key difference is that we re-sign the summands only once, rather than multiple times. Like in \cite{Truemper2016}, we start with totally unimodular signings exhibiting regularity of the two summands. Then we multiply their rows and columns by $\pm 1$ factors (which preserves total unimodularity) so that the submatrix $
\begin{NiceArray}{ccc}[hvlines]
    1 & 1 & 0 \\
    \Block[draw]{2-2}{D_{0}} & & 1 \\
    & & 1 \\
\end{NiceArray}
$ is signed in both summands simultaneously as either $
\begin{bmatrix}
    1 & 1 & 0 \\
    1 & 0 & 1 \\
    0 & -1 & 1 \\
\end{bmatrix}
$ or $
\begin{bmatrix}
    1 & 1 & 0 \\
    1 & 1 & 1 \\
    0 & 1 & 1 \\
\end{bmatrix}
$, depending on whether $D_{0}$ is $\begin{bmatrix} 1 & 0 \\ 0 & 1 \\ \end{bmatrix}$ or $\begin{bmatrix} 1 & 0 \\ 1 & 1 \\ \end{bmatrix}$. Thus, we get totally unimodular signings of the summands that coincide on the intersection, which allows us to define the \emph{canonical} signing of the entire 3-sum: use the same signs as in the re-signed summands everywhere except for the bottom-left block, which is signed via $D_{\ell r}' = D_{r}' \cdot (D_{0}')^{-1} \cdot D_{\ell}'$, and the $0$ block, which remains as is.

% Only one re-signing: advantage
The main advantage of our approach is that we avoid chained constructions and proofs of properties of such constructions, and we do not need to define $\Delta$-sums. Moreover, unlike \cite{Truemper2016}, our proof does not rely on the general lemma about re-signing totally unimodular matrices. This detail is crucial, as the proof of this lemma in \cite{Truemper2016} involves a graph-theoretic argument, which would be very challenging to formalize in Lean with the current tools available in Mathlib.

% Well-behaved family of matrices
The second major difference from the approach of \cite{Truemper2016} is that our main argument does not deal with signings of 3-sums directly. Instead, we work with a matrix family called \texttt{MatrixLikeSum3} in our code. This allows us to split the proof of regularity of 3-sums into three clear steps. First, we show that pivoting on a non-zero entry in the top-left block of any matrix from this family produces a matrix that also belongs to this family. Next, we utilize the result from the first step to prove that every matrix in this family is totally unimodular. We do this via a similar argument to the proof that 2-sums of totally unimodular matrices are totally unimodular. Finally, we show that every canonical signing of the 3-sum matrix defined above is included in this matrix family and is thus totally unimodular. Overall, this proof takes a more systematic approach to deriving properties of signings of 3-sums and using them to prove their total unimodularity. Additionally, it conveniently reuses a large portion of the argument for 2-sums.

% Ivan's version; this paragraph seems optional
In some proofs, we worked with large case splits with up to 896 cases. To handle such situations, we used \texttt{all\_goals try} followed by one or more tactics, discharging multiple goals at once without selecting them by hand or repeating the proof. We repeatedly applied this method to discharge the remaining goals in waves until the proof was complete.
