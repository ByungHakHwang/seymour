\section{Proof of Regularity}

% Final theorems
In our library, the final theorems that regularity is preserved under 1-, 2-, and 3-sums are stated as follows.
% Discussion: \url{https://leanprover.zulipchat.com/\#narrow/channel/113488-general/topic/How.20to.20best.20recall.20P.20.E2.88.A7.20Q.20.E2.86.92.20R/with/532666812
\begin{leancode}
theorem Matroid.IsSum1of.isRegular {α : Type*}
    [DecidableEq α] {M Mₗ Mᵣ : Matroid α} :
  M.IsSum1of Mₗ Mᵣ → M.RankFinite →
    Mₗ.IsRegular → Mᵣ.IsRegular → M.IsRegular
\end{leancode}
\begin{leancode}
theorem Matroid.IsSum2of.isRegular {α : Type*}
    [DecidableEq α] {M Mₗ Mᵣ : Matroid α} :
  M.IsSum2of Mₗ Mᵣ → M.RankFinite →
    Mₗ.IsRegular → Mᵣ.IsRegular → M.IsRegular
\end{leancode}
\begin{leancode}
theorem Matroid.IsSum3of.isRegular {α : Type*}
    [DecidableEq α] {M Mₗ Mᵣ : Matroid α} :
  M.IsSum3of Mₗ Mᵣ → M.RankFinite →
    Mₗ.IsRegular → Mᵣ.IsRegular → M.IsRegular
\end{leancode}
\todo[inline]{@Ivan: review the paragraph below}
This design has several advantages.
All assumptions meet on the level of matroids.
Matroid is the central notion.
It is therefore much more interesting than if we proved properties of their representations only.
It has also practical advantages.
The user of our library can provide matroids $M_{\ell}$, $M_{r}$, and $M$, and they
can use three representations for witnessing that $M$ is a 3-sum of $M_{\ell}$ and $M_{r}$,
a different representation for witnessing that $M$ has a finite rank,
and different representations for witnessing that $M_{\ell}$ and $M_{r}$ are regular.

% High-level breakdown
We split the proof of each of these theorems into three stages corresponding to the three abstraction layers used for definitions: the matroid level, the standard representation level, and the matrix level. The final matroid-level theorems are reduced to the respective lemmas for standard representations by applying \texttt{StandardRepr.toMatroid\_isRegular\_iff\_hasTuSigning} and \texttt{StandardRepr.finite\_X\_of\_toMatroid\_rankFinite} in all three proofs (for the 1-, 2-, and 3-sums). The reduction from the standard representation level to the matrix level for 1- and 2-sums is straightforward\.---\.plug the standard representation matrices and their (rational) signings into \texttt{matrixSum1} and \texttt{matrixSum2}, respectively. For 3-sums, this reduction is more involved, as we additionally apply the following lemma to simplify the assumption on $D_{0}$:
\begin{leancode}
lemma Matrix.isUnit_2x2
    (A : Matrix (Fin 2) (Fin 2) Z2)
    (hA : IsUnit A) :
  ∃ f : Fin 2 ≃ Fin 2,
  ∃ g : Fin 2 ≃ Fin 2,
    A.submatrix f g = 1 ∨
    A.submatrix f g = !![1, 1; 0, 1]
\end{leancode}
In short, either $D_{0} = \begin{bmatrix} 1 & 0 \\ 0 & 1 \\ \end{bmatrix}$ or $D_{0} = \begin{bmatrix} 1 & 0 \\ 1 & 1 \\ \end{bmatrix}$, up to reindexing. Performing the reduction at this stage allows us to simplify the code by only performing the case distinction between the two forms of $D_{0}$ without the need to invoke \texttt{Matrix.isUnit\_2x2} every time.

% Matrix-level Implementation for 1-Sums
For 1-sums, the matrix-level result is captured by
\begin{leancode}
lemma Matrix.fromBlocks_isTotallyUnimodular
    {X₁ X₂ Y₁ Y₂ R : Type*}
    [LinearOrderedCommRing R]
    [DecidableEq X₁] [DecidableEq X₂]
    [DecidableEq Y₁] [DecidableEq Y₂]
    {A₁ : Matrix X₁ Y₁ R} {A₂ : Matrix X₂ Y₂ R}
    (hA₁ : A₁.IsTotallyUnimodular)
    (hA₂ : A₂.IsTotallyUnimodular) :
    (Matrix.fromBlocks A₁ 0 0 A₂
    ).IsTotallyUnimodular
\end{leancode}

% Matrix-level Implementation for 2-Sums
While our proof that 1-sums preserve total unimodularity of matrices closely follows \cite{Truemper2016}, our proof for 2-sums is simpler and more streamlined as a result of several changes. The original proof in \cite{Truemper2016} is by contradiction, frames the argument as an iterative procedure, and works with so-called minimum violation matrices. In contrast, we have an inductive proof of the positive statement utilizing the notion of partially unimodular matrices, which are defined as the weaker version of totally unimodular matrices:
\begin{leancode}
def Matrix.IsPartiallyUnimodular
    (A : Matrix X Y R) (k : ℕ) : Prop :=
  ∀ f : Fin k → X, ∀ g : Fin k → Y,
    (A.submatrix f g).det ∈ SignType.cast.range
\end{leancode}
Note that if a matrix is $k$-partially unimodular for every $k \in \mathbb{N}$, then it is totally unimodular. Since a 2-sum of totally unimodular matrices has entries in $\{0, \pm 1\}$, it is $1$-partially unimodular, which serves as the base for our inductive argument. We then show that if the 2-sum of two totally unimodular matrices is $k$-partially unimodular for some $k$, then it is also $(k + 1)$-partially unimodular. The proof of the inductive step involves a case analysis and showing that the matrix shape of a 2-sum is preserved under pivoting, adapting a similar argument from \cite{Truemper2016} to the new proof structure. Thanks to these changes, the proof of regularity of 2-sums takes up about one page in our blueprint and a few hundred lines of code in our implementation.


% Matrix-level Implementation for 3-Sums
For 3-sums, the matrix-level proof was significantly reworked to simplify and streamline the approach of \cite{Truemper2016}. On a high level, there are two major differences between our proof and the original approach. The first important change is that instead of re-signing the summands of the 3-sum several times as it is done in \cite{Truemper2016}, we perform the re-signing only once. To this end, we start with totally unimodular signings exhibiting regularity of the two summands, and we multiply their rows and columns by $\pm 1$ factors, preserving total unimodularity, so that the submatrix $
\begin{NiceArray}{ccc}[hvlines]
    1 & 1 & 0 \\
    \Block[draw]{2-2}{D_{0}} & & 1 \\
    & & 1 \\
\end{NiceArray}
$ has the same signing in both summands. In fact, we can get the signing of this submatrix to become either $
\begin{bmatrix}
    1 & 1 & 0 \\
    1 & 0 & 1 \\
    0 & -1 & 1 \\
\end{bmatrix}
$ or $
\begin{bmatrix}
    1 & 1 & 0 \\
    1 & 1 & 1 \\
    0 & 1 & 1 \\
\end{bmatrix}
$, depending on whether $D_{0}$ is $\begin{bmatrix} 1 & 0 \\ 0 & 1 \\ \end{bmatrix}$ or $\begin{bmatrix} 1 & 0 \\ 1 & 1 \\ \end{bmatrix}$. As a result, we get totally unimodular signings of the summands that coincide on the intersection, which allows us to define the \emph{canonical} signing of the entire 3-sum by using the same signs as in the re-signed everywhere except for the bottom-left block, which we sign via $D_{\ell r}' = D_{r}' \cdot (D_{0}')^{-1} \cdot D_{\ell}'$, and the $0$ block, which remains as is.

The second major difference from the approach of \cite{Truemper2016} is that we introduce the following matrix family:
% Martin: I would rather omit the entire code snippet below.
\begin{leancode}
structure MatrixLikeSum3
    (Xₗ Yₗ Xᵣ Yᵣ : Type*)
    (c₀ c₁ : Fin 2 ⊕ Xᵣ → ℚ) where
  Aₗ : Matrix Xₗ Yₗ ℚ
  D  : Matrix (Fin 2 ⊕ Xᵣ) Yₗ ℚ
  Aᵣ : Matrix (Fin 2 ⊕ Xᵣ) Yᵣ ℚ
  LeftTU : (Aₗ ⊟ D).IsTotallyUnimodular
  /-- omitted: further properties -/
\end{leancode}
This allows us to split the proof of the desired result about 3-sums into three steps. First, we show that pivoting on a non-zero entry in the $A_{\ell}$ block of any matrix from the \texttt{MatrixLikeSum3} family produces a matrix that also belongs to this family. Next, we utilize the previous result to prove that every matrix in this family is totally unimodular via a similar argument to the proof that 2-sums of totally unimodular matrices are totally unimodular. Finally, we show that every canonical signing of the 3-sum matrix defined above belongs to this matrix family and thus is totally unimodular.

Our implementation of the proof for 3-sums significantly simplifies the proof strategy of \cite{Truemper2016}. Defining the canonical signing for the 3-sum matrix with just one resigning of the summands allows us to avoid chained constructions and helps decouple proofs of properties of the resulting objects from the construction process. Our implementation of the re-signing also avoids the lemma about re-signings of general totally unimodular matrices, whose proof in \cite{Truemper2016} relies on a graph-theoretic argument, which would be very challenging to formalize in Lean with the current tools available in Mathlib. Moreover, our approach does not require defining $\Delta$-sums or proving their properties. Finally, working with the family of 3-sum-like matrices allows us to systematize the properties of signings of 3-sums used to prove their total unimodularity and have the implementation of the proof resemble that of the 2-sum case as closely as possible.

\todo[inline]{@Ivan: review the paragraph below}
In some proofs, we were working with large case splits (bruteforcing over 100 cases).
Usually, most of the cases were trivial to resolve but some cases required extra work.
Lean was really helpful in these proofs.
We could use \texttt{all\_goals try} followed by a tactic or a block of tactics that discharged the easy goals
without having to delimit ahead of time which of the goals were the easy ones.
Afterwards, we addressed the remaining cases by a different tactic block, sometimes with multiple fallbacks.
