\subsection{Totally Unimodular Matrices}

In our work, regular matroids are defined in terms of totally unimodular matrices \cite{Oxley2011,Truemper2016}. Before introducing their definition, let us review how matrices and submatrices are implemented in Mathlib. A matrix with rows indexed by \texttt{m}, columns indexed by \texttt{n}, and entries of type \texttt{α} is represented by \texttt{Matrix m n α}, implemented as a (curried) binary function \leaninline{m → n → α}. Thus, the elements of matrix \texttt{A} can be accessed with \texttt{A i j}. Similarly, \texttt{Matrix.submatrix} is defined so that \texttt{(A.submatrix f g) i j = A (f i) (g j)} holds. For example, if
\[
    A = \begin{bmatrix}
       1 & 2 & 3 \\
       4 & 5 & 6 \\
       7 & 8 & 9
    \end{bmatrix},
    \quad
    \texttt{f = ![0]}\,,
    \quad
    \texttt{g = ![2, 2, 0, 0]}\,,
\]
then
$
    \texttt{A.submatrix f g} =
    \begin{bmatrix}
       3 & 3 & 1 & 1
    \end{bmatrix}
$, typed as a matrix, not a vector. Note that \texttt{Matrix.submatrix} may repeat and reorder rows and columns.

Now, a matrix $A$ over a commutative ring $R$ is called \emph{totally unimodular} if every finite square submatrix of $A$ (not necessarily contiguous, with no row or column taken twice) has determinant in $\{-1, 0, 1\}$. Mathlib implements this definition as follows:
\begin{leancode}
def Matrix.IsTotallyUnimodular {m n R : Type*}
    [CommRing R] (A : Matrix m n R) :
    Prop :=
  ∀ k : ℕ, ∀ f : Fin k → m, ∀ g : Fin k → n,
    f.Injective → g.Injective →
      (A.submatrix f g).det ∈
        Set.range SignType.cast
\end{leancode}
Here, \texttt{SignType} is an inductive type with three values: \texttt{zero}, \texttt{neg}, and \texttt{pos}; and \texttt{SignType.cast} maps them to \texttt{(0 : R)}, $\texttt{(-1 : R)}$, and $\texttt{(1 : R)}$, respectively.

Also note that the indexing functions \texttt{f} and \texttt{g} are required to be injective in the definition, but this condition can be lifted. Indeed, lemma \texttt{Matrix.isTotallyUnimodular\_iff} shows that one can equivalently check the determinants of all finite square submatrices, not just ones with no repeated rows or columns.

Keep in mind that the determinant is computed over $R$, so for certain commutative rings, all matrices are trivially totally unimodular, for example, for $R = \mathbb{Z}_{3}$.
